\chapter*{Resumen}
La Universidad Nacional de Quilmes dispone de una gran cantidad de datos que fue recabando a lo largo de su historia. Los alumnos, las materias, la carreras, los planes de estudios, notas, inscripciones, postulantes, entre otros. 
Algunas carreras, como las relacionadas a Informática fueron pidiendo esos datos y generando planillas para luego analizar esos datos. Las direcciones de las carreras trabajan alrededor de una semana a tiempo completo post inscripciones y terminadas las cursadas para hacer un procesamiento adecuado con  la información de los estudiantes (sobretodo aquellos que comenzaron el Ciclo Básico de las carreras). Durante el cuatrimestre, estos registros se consultan constantemente para generar diferentes estadísticas para planificar la evolución de la carrera. Aún cuando toda la información está disponible y analizada, todo este proceso es ineficiente y demora decisiones relevantes a la carrera, en lo que respecta tanto a estudiantes como a docentes.
El objetivo que tiene  esta tésis es el de analizar las diferentes necesidades de las direcciones de carreras de la UNQ con respecto a los datos, para luego desarrollar un sistema donde estos datos puedan ser consultados, analizados y visualizados de forma gráfica.