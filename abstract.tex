\chapter*{Resumen}
Dentro del contexto universitario, las universidades nacionales generan a lo largo de su historia diferentes repositorios de datos. Éstos están relacionados a las carreras (desde su creación) y sus respectivos planes de estudios, las diferentes materias y profesores que las dictan en esas carreras y como datos fundamentales, todos los datos de los estudiantes (desde su ingreso), de sus inscripciones y sus notas.

La Universidad Nacional de Quilmes está enmarcada dentro de este contexto. Particularmente desde la creación de las carreras Tecnicatura en Programación Informática y la Licenciatura en Informática, las direcciones de ambas carreras han obtenido todos los datos relacionados a sus estudiantes y han generado diferentes bases de datos \"ad-hoc\" para realizar diferentes análisis iniciales de la población de ambas carreras.

Éstos análisis se realizan tanto durante el proceso de inscripciones como durante el transcurso de cada cuatrimestre.

Debido a la falta de una sistematización del uso de datos, aún cuando toda la información está disponible y analizada, todo este proceso de análisis es ineficiente y demora decisiones relevantes a la carrera, en lo que respecta tanto a estudiantes como a docentes.
El objetivo que de este trabajo es el de analizar las diferentes necesidades de las direcciones de carreras de la UNQ con respecto a los datos, para luego desarrollar un sistema donde estos datos puedan ser consultados y analizados con mayor facilidad.