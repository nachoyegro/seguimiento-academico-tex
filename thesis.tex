\documentclass[12pt,letterpaper,final,oneside,openany,onecolumn]{book} 												
\usepackage[lmargin=3.5cm,rmargin=2.5cm,tmargin=3.0cm,bmargin=3.0cm]{geometry}
%\usepackage[latin1]{inputenc}											%european 
\usepackage[T1]{fontenc}
\usepackage[]{times}
\usepackage[spanish]{babel}													%español
\usepackage{amsmath}	
\usepackage{bera}% optional: just to have a nice mono-spaced font
\usepackage{listings}
\usepackage{xcolor}
\usepackage{graphicx}
\usepackage{multicol}
\usepackage{url}
\usepackage{hyperref}
\usepackage[utf8]{inputenc}
\usepackage{amssymb}
\usepackage{outlines}
\usepackage{array}
%\usepackage{algorithm_spa}
\usepackage{algpseudocode}
\usepackage[footnotesize]{subfigure}
\usepackage{makeidx}
\usepackage{color}
\usepackage{caption}
\makeindex
%\renewcommand{\baselinestretch}{1}
%\renewcommand{\contentsname}{Índice General}%{Tabla de Contenidos}
%\renewcommand{\listfigurename}{Lista de Figuras}
%\renewcommand{\listtablename}{\'Indice de Tablas}
%\renewcommand{\chaptername}{Capítulo}
%\renewcommand{\bibname}{Bibliografía}
\renewcommand\floatpagefraction{.9}
\renewcommand\topfraction{.9}
\renewcommand\bottomfraction{.9}
\renewcommand\textfraction{.1}
\addto\captionsspanish{%
  \renewcommand{\tablename}%
    {Tabla}%
}

\lstdefinelanguage{json}{
    basicstyle=\normalfont\ttfamily,
    numbers=left,
    numberstyle=\scriptsize,
    stepnumber=1,
    numbersep=8pt,
    showstringspaces=false,
    breaklines=true,
    frame=lines,
    backgroundcolor=\color{background},
    literate=
     *{0}{{{\color{numb}0}}}{1}
      {1}{{{\color{numb}1}}}{1}
      {2}{{{\color{numb}2}}}{1}
      {3}{{{\color{numb}3}}}{1}
      {4}{{{\color{numb}4}}}{1}
      {5}{{{\color{numb}5}}}{1}
      {6}{{{\color{numb}6}}}{1}
      {7}{{{\color{numb}7}}}{1}
      {8}{{{\color{numb}8}}}{1}
      {9}{{{\color{numb}9}}}{1}
      {:}{{{\color{punct}{:}}}}{1}
      {,}{{{\color{punct}{,}}}}{1}
      {\{}{{{\color{delim}{\{}}}}{1}
      {\}}{{{\color{delim}{\}}}}}{1}
      {[}{{{\color{delim}{[}}}}{1}
      {]}{{{\color{delim}{]}}}}{1},
}
\colorlet{punct}{red!60!black}
\definecolor{background}{HTML}{EEEEEE}
\definecolor{delim}{RGB}{20,105,176}
\colorlet{numb}{magenta!60!black}

\setcounter{totalnumber}{50}
\setcounter{topnumber}{50}
\setcounter{bottomnumber}{50}
\setcounter{secnumdepth}{3}

% Different font in captions
\newcommand{\captionfonts}{\small}

\makeatletter  % Allow the use of @ in command names
\long\def\@makecaption#1#2{%
  \vskip\abovecaptionskip
  \sbox\@tempboxa{{\captionfonts #1: #2}}%
  \ifdim \wd\@tempboxa >\hsize
    {\captionfonts #1: #2\par}
  \else
    \hbox to\hsize{\hfil\box\@tempboxa\hfil}%
  \fi
  \vskip\belowcaptionskip}
\makeatother   % Cancel the effect of \makeatletter

%Inicio de documento
\begin{document}
\hyphenation{pa-la-bra}

\renewcommand{\contentsname}{Tabla de Contenidos}
%\renewcommand{\listfigurename}{Lista de Figuras}
%\renewcommand{\listtablename}{Lista de Tablas}
	\frontmatter
		\label{ch:portada}
\thispagestyle{empty}

%\vspace{-2.0cm}
	\begin{figure}[t]
						\centering
							\includegraphics[height=0.15\textwidth]{./images/logoUCV.jpg}
	\end{figure}
					\begin{center}
						Universidad Central de Venezuela\\
						Facultad de Ciencias\\
						Escuela de Computaci\'on\\
						Centro de Computaci\'on Gr\'afica\\
					\end{center}
					
					\vspace{2.5cm}
					\begin{center}
						\large{\textbf{T\'itulo de la tesis}}
					\end{center}
					
					\vspace{5.2cm}
					\begin{center}
						Trabajo Especial de Grado \\
						presentado ante la Ilustre\\
						Universidad Central de Venezuela\\
						Por el (o los) Bachiller (es)\\
						Mark Hamill\\
						para optar el título de \\
						Licenciado en Computaci\'on
					\end{center}
					
					\begin{center}
						Tutor: Prof. Luke Skywalker\\
					\end{center}
					
					\vspace{1.0cm}
					\begin{center}
						Caracas, Diciembre de 2020
					\end{center}
						
					
%\newpage
		%\input{agradecimientos}	%opcional
		%\input{dedicatoria}		%opcional
		\chapter*{Resumen}
Dentro del contexto universitario, las universidades nacionales generan a lo largo de su historia diferentes repositorios de datos. Éstos están relacionados a las carreras (desde su creación) y sus respectivos planes de estudios, las diferentes materias y profesores que las dictan en esas carreras y como datos fundamentales, todos los datos de los estudiantes (desde su ingreso), de sus inscripciones y sus notas.

La Universidad Nacional de Quilmes está enmarcada dentro de este contexto. Particularmente desde la creación de las carreras Tecnicatura en Programación Informática y la Licenciatura en Informática, las direcciones de ambas carreras han obtenido todos los datos relacionados a sus estudiantes y han generado diferentes bases de datos \"ad-hoc\" para realizar diferentes análisis iniciales de la población de ambas carreras.

Éstos análisis se realizan tanto durante el proceso de inscripciones como durante el transcurso de cada cuatrimestre.

Debido a la falta de una sistematización del uso de datos, aún cuando toda la información está disponible y analizada, todo este proceso de análisis es ineficiente y demora decisiones relevantes a la carrera, en lo que respecta tanto a estudiantes como a docentes.
El objetivo que de este trabajo es el de analizar las diferentes necesidades de las direcciones de carreras de la UNQ con respecto a los datos, para luego desarrollar un sistema donde estos datos puedan ser consultados y analizados con mayor facilidad.
		\chapter{Abreviaturas}
UNQ
API
REST
JWT
VM
JSON
UI
JWS
WE,

		\tableofcontents	%indice
		\listoffigures	%opcional
		\listoftables		%opcional
	\mainmatter
		\chapter{Introducción}
\label{sec:introduccion}


El déficit en ingenieros y graduados en las llamadas disciplinas STEM (Ciencia, Tecnología, Ingeniería y Matemática) es un problema que no sólo afecta a nuestro país, sino que se manifiesta como una problemática global. Particularmente en Argentina, cada año quedan sin cubrir 5.000 puestos en la industria del software por falta de profesionales (según la Cámara de la Industria Argentina del Software). El sector emplea a 90.000 personas y representa una de las principales exportaciones de valor agregado con un crecimiento del 10\% anual, pero “la matrícula en carreras de sistemas quedó estancada en 20.000 y se reciben 4.000 por año, cuando la industria requiere el doble“ (según Fundación Sadosky).
Bajo este contexto, se implementaron la Tecnicatura en Programación Informática en 2003 y la Licenciatura en Informática en 2012 en la Universidad Nacional de Quilmes para desarrollar profesionales altamente calificados y ofrecer carreras distintivas de aquellas ofertadas por los centros universitarios próximos (UBA, UNLP, etc) y que apunten a cubrir necesidades sociales e industriales concretas.
 
Desde la creación de ambas carreras, se ha dado un crecimiento constante  en sus respectivas matrículas. A partir del 2015, ese crecimiento es más notable en la UNQ pues se incorporaron a sus planes de estudio las materias del Ciclo Ingreso como parte de las materias de la carrera, denominándolo como Ciclo Introductorio. A partir de este cambio, los estudiantes ya eran parte de la carrera cuando se incorporaban a la UNQ, lo que no ocurría en años previos. Los números del crecimiento se pueden ver en las siguientes tablas:


\begin{table}[!htbp]
    \centering
    \begin{tabular}{|c|c|c|c|}
    \hline
    Año & Estudiantes & Total Matrícula & Tasa Crecimiento \\
    \hline
    2013 & 23 & 23 & \\
    \hline
    2014 & 94 & 117 & 5,09 \\ 
    \hline
    2015 & 97 & 214 & 1,83 \\
    \hline
    2016 & 333 & 547 & 2,56 \\
    \hline
    2017 & 323 & 870 & 1,59 \\ 
    \hline
    2018 & 297 & 1167 & 1,34 \\ 
    \hline
    \end{tabular}
    \caption{Crecimiento de la Matrícula en la Licenciatura en Informática}
    \label{tab:tabla_planes}
\end{table}



\begin{table}[!htbp]
    \centering
    \begin{tabular}{|c|c|c|c|}
    \hline
    Año & Estudiantes & Total Matrícula & Tasa Crecimiento \\
    \hline
    2003 & 1 & 1 & \\
    \hline
    2004 & 3 & 4 & 4,00 \\
    \hline
    2005 & 4 & 8 & 2,00 \\
    \hline
    2006 & 0 & 8 & 1,00 \\
    \hline
    2007 & 41 & 49 & 6,13 \\
    \hline
    2008 & 106 & 155 & 3,16 \\
    \hline
    2009 & 99 & 254 & 1,64 \\
    \hline
    2010 & 108 & 362 & 1,43 \\
    \hline
    2011 & 142 & 504 & 1,39 \\ 
    \hline
    2012 & 112 & 616 & 1,22 \\
    \hline
    2013 & 166 & 782 & 1,27 \\
    \hline
    2014 & 103 & 885 & 1,13 \\
    \hline
    2015 & 88 & 973 & 1,10 \\
    \hline
    2016 & 287 & 1260 & 1,29 \\
    \hline
    2017 & 295 & 1555 & 1,23 \\
    \hline
    2018 & 271 & 1826 & 1,17 \\
    \hline
    \end{tabular}
    \caption{Crecimiento de la Matrícula en la Tecnicatura en Programación Informática}
    \label{tab:tabla_planes}
\end{table}

Desde sus inicios, la dirección de ambas carreras ha creado y mantenido un registro de datos personales de los estudiantes, las materias cursadas y las materias que tienen intenciones de cursar cuatrimestre a cuatrimestre. Con este registro, inicialmente se ha permitido organizar las intenciones de inscripción de cada estudiante, generar las listas de emails con las cuales trabaja cada materia, manejar el cupo de las comisiones de las materias para evitar superpoblación en las mismas, detectar posibles candidatos a presentarse como auxiliares académicos, detectar diferentes problemáticas en la cursada de los estudiantes (sobretodo en las instancias finales de la carrera) y realizar diferentes análisis de la evolución de la carrera de los estudiantes durante cada cuatrimestre, entre otras.
En la actualidad, todos los análisis descritos previamente se realizan en base a planillas Excel que mantienen toda la información de los estudiantes y materias. Esta información ha sido obtenida ya sea a través de los estudiantes mismos o bien, usando el sistemas SIU-Guaraní de la UNQ.
Con el incremento de la matrícula año a año, este proceso de análisis “ad-hoc” se hace más complejo y lleva demasiado tiempo. Las direcciones de las carreras de informática trabajan alrededor de una semana a tiempo completo post inscripciones y terminadas las cursadas para hacer un procesamiento adecuado con  la información de los estudiantes (sobretodo aquellos que comenzaron el Ciclo Básico de cualquiera de las carreras). Durante el cuatrimestre, el sistema de registros se consulta constantemente para generar diferentes estadísticas para planificar la evolución de la carrera. Aún cuando toda la información está disponible y analizada, todo este proceso es ineficiente y demora decisiones relevantes a la carrera, en lo que respecta tanto a estudiantes como a docentes. Los problemas actuales (entre otros) son:
Como la información se guarda en planilla “Excel”, no existe la posibilidad de tener varios usuarios (directores, asistentes de carrera, docentes) colaborando en forma cooperativa (distribuida).
Los datos provistos por SIU-Guaraní y los resultados de cada materia al finalizar cada cuatrimestre se incorporan de forma manual en los registros.
Los análisis se realizan con fórmulas analizando las planillas y su información (ejemplo, generación de estadísticas)
No existe una forma automática de controlar la información que tiene SIU-Guaraní acerca de los estudiantes y su situación de sus cursadas en las carreras.

\section[Objetivo general]{Objetivo general}
En base a esta lista de problemas que se enumeran en la Introducción, este plan de trabajo plantea como objetivo general el desarrollo de una aplicación de seguimiento académico que puede ayudar en el proceso de gestión y decisión de la Dirección de Carreras de TPI/LI, con las siguientes acciones:
Analizar información de los estudiantes y de las materias que cursa o debe cursar de la TPI y de la LI, desde su incorporación a la carrera hasta el final de la carrera    
Generar recomendaciones para los estudiantes acerca de sus elecciones de materia a principios de cada cuatrimestre
Generar estadísticas y visualizaciones relevantes que ayuden a la Dirección de Carreras a tomar decisiones durante el año académico.

\section[Objetivos específicos]{Objetivos específicos}
Para el desarrollo de la aplicación de seguimiento académico, los objetivos específicos en base a la implementación de la plataforma serán los siguientes:

\begin{outline}
    \1 Relevamiento de los requerimientos y casos de uso.
    \1 Relevamiento del hardware disponible en la Universidad.
    \1 Diseño del software y la arquitectura del mismo.
    \1 Desarrollo de software.
    \1 Tests de unidad e integración.
    \1 Pruebas de stress.
    \1 Documentación.
    \1 Puesta en producción.
\end{outline}
		\chapter{Estado del arte}
\label{sec:hello}

\section[Crecimiento exponencial de los datos]{Crecimiento exponencial de los datos} 


La humanidad, hasta el año 2005, creó 130 ExaBytes de datos (un exabyte son 1.073.741.824 GB). Para el año 2010, la cantidad de datos subió a 1200 exabytes. Para el año 2015, la cantidad de datos subió a 7900 exabytes. Y para el año 2020, se estima que esa cantidad llegó a los 40900 exabytes. [[14]]

En resumen, la cantidad de datos que producimos los humanos está creciendo exponencialmente, y las universidades no son la excepción.



\section[El papel de las universidades con respecto a los datos]{El papel de las universidades con respecto a los datos}

Hoy en dia, se está transformando más en una norma que en una excepción para las universidades del mundo utilizar los datos que tienen sobre los estudiantes para propósitos académicos. Hay universidades que usan estos datos para ayudar a los estudiantes a tener éxito, o para ofrecerles ayuda cuando se considera que no están rindiendo, o bien para mantenerlos cursando y seguir recibiendo ingresos, en el caso de las privadas.
El análisis de los datos se hace cada vez más frecuentementemente, y de manera más eficiente ya que las herramientas de análisis fueron mejorando.

\section[Universidad de San Franciso]{Universidad de San Franciso}

En la Universidad de San Frascisco (USF), los datos de los estudiantes tales como asistencia, notas, y materias son usados para determinar su progreso [[12]]. Cuando ven que un alumno está teniendo dificultades, usando la tecnología desarrollada por el departamento de sistemas de USF, alertan a los tutores para que hablen con el estudiante [[12]].
Además, usan una aplicación móvil llamada USFMobile, que consulta datos de sus sistemas y generan alertas.
Eventualmente, esos datos serán usados para "entender qué hacen los alumnos, y tratar de ayudarlos" [[12]].

\section[Universidad de Missouri]{Universidad de Missouri}
La Universidad de Missouri arma cuestionarios de 12 preguntas para los alumnos. El propósito es conocer cómo los estudiantes se desenvuelven en términos académicos, sociales y financieros. En caso que un alumno indique alguna inquietud, esa información es utilizada para tener un mayor impacto [[13]]. Los datos son almacenados en los sitemas de la universidad.
Los tres temas principales que mostró la encuesta, estan relacionados con las dificultades de algunos cursos, la asistencia y las dificultades financieras de los alumnos. Estos datos le dieron a la Universidad un panorama y pudo actuar rápidamente en consecuencia. [[13]]

\section[Universidad Estatal de Georgia]{Universidad Estatal de Georgia}

La Universidad Estatal de Georgia se convirtió en un ejemplo a nivel nacional en Estados Unidos, por el papel que adaptó para ayudar a los estudiantes de bajos recursos y los de primera generación. 
Esta universidad usa análisis predictivos para identificar a los alumnos que podrían abandonar o no llegar a recibirse. Esto lo logran usando información pasada, la cual es utilizada para predecir futuros eventos. 
La Universidad de Georgia analizó 2 millones y medio de calificaciones obtenidas por los estudiantes a lo largo de 10 años para crear una lista de factores que predecían qué alumnos son menos probables a graduarse. Este sistema tiene más de 800 alertas, apuntadas a indicarle a los tutores acerca de sus estudiantes y así poder ayudarlos. Por ejemplo, un tutor recibe un alerta cuando un estudiante no recibe una nota satisfactoria en una materia que es fundamental. 





\section[Trabajos relacionados]{Trabajos relacionados}

\subsection[eCoach]{eCoach}
ECoach es una aplicación diseñada en la Universidad de Michigan para estudiantes de primer año que toman clases de ciencia, tecnología, ingeniería o matemática. Esta aplicación toma datos que las universidades ya tienen, como las materias que cursó un alumno, las notas, etc.
Otro aspecto importante de esta aplicación, es que se recibe *feedback* de parte de los alumnos con respecto a materias, profesores, horarios, etc.


\section[El papel de la Universidad Nacional de Quilmes]{El papel de la Universidad Nacional de Quilmes}

La Universidad Nacional de Quilmes tiene en la actualidad alrededor de , y a lo largo de su historia tuvo . Esos alumnos cursaron materias, tuvieron notas, y dejaron información que no está siendo explotada.
Si un director de una carrera determinada, quisiera tener los datos de sus alumnos, tiene que hacer un pedido formal y esos datos se les son entregados en forma de planilla. Esos datos no están normalizados, y no existe un estándar. Es decir, un año vienen de una forma, y otro año de otra forma diferente.

\section[Planillas de datos]{Planillas de datos}

Las planillas que se le son entregadas a las carreras de la Universidad Nacional de Quilmes a través de un pedido son las siguientes.

\subsection[Planes de estudio]{Planes de estudio}

Los datos de planes de estudio traen información de las materias junto con el área a la que pertenecen, el núcleo, y los créditos, entre otros.
Las columnas de la planilla es de la siguiente forma:

\begin{table}[]
    \centering
    \begin{tabular}{|c|c|c|c|c|c|c|c|}
    \hline
    Carrera & Plan & Cuat. & Núcleo & Área & Materia & Créd. & Nombre \\ \hline
    W & 2019 & 3 & B & Programación & 1035 & 12 & Base de Datos  \\
    \hline
    \end{tabular}
    \caption{Ejemplo de una planilla de datos de un plan de estudios}
    \label{tab:tabla_planes}
\end{table}

Con esta información, alcanzaría para tener los datos de las carreras, los planes y las materias.

\subsection[Datos personales]{Datos personales}

La planilla de datos personales trae la siguiente información en orden: 

\begin{table}[]
    \centering
    \begin{tabular}{|c|c|c|c|c|c|c|c|}
    \hline
    Legajo & DNI & Apellido & Nombre & Email & Fecha & Carrera & Plan \\ \hline
    21872 & 35905769 & Yegro & Juan & jy@unq.edu.ar & 01-02-2009 & W & 2019 \\
    \hline
    \end{tabular}
    \caption{Ejemplo de una planilla de datos personales de los alumnos}
    \label{tab:tabla_datos}
\end{table}

Estos datos son de ayuda para que los directores puedan identificar a los alumnos.

\subsection[Materias cursadas]{Materias cursadas}

Esta es, quizás, la planilla más cargada de datos ya que trae el historial de cursadas de una carrera determinada. Cada fila representa la cursada de un alumno en una materia determinada.


\begin{outline}
    \2 Legajo.
    \2 DNI 
    \2 Carrera 
    \2 Regular 
    \2 Calidad 
    \2 Materia 
    \2 Nombre 
    \2 Fecha 
    \2 Resultado 
    \2 Nota 
    \2 Forma 
    \2 Credito 
    \2 Acta 
    \2 Acta E. 
    \2 Plan
\end{outline}

\subsection[Inscripciones]{Inscripciones}

Esta planilla indica las inscripciones a una materia por parte de un alumno.

\begin{table}[]
    \centering
    \makegapedcells
    \begin{tabular}{|c|c|c|c|c|c|}
    \hline
    Carrera & DNI & Legajo & Código de materia & Comision & Fecha  \\\hline
    W & 35905769 & 21872 & 01036 & C2 & 01/03/2016  \\
    \hline
    \end{tabular}
    \caption{Ejemplo de una planilla de inscripciones de alumnos}
    \label{tab:tabla_datos}
\end{table}


\section[Análisis de requerimientos]{Análisis de requerimientos}

Luego de hacer un análisis de requerimientos con directores de las distinas carreras de la Universidad Nacional de Quilmes, se llegó a que las principales necesidade son:

\begin{outline}
    \1 Se necesita que puedan calcularse el *avance de carrera*, *avance de diplomatura* y *avance de ciclo introductorio* en base a los créditos.
    \1 Se necesita que puedan calcularse el porcentaje de materias aprobadas del total de la Carrera, el porcentaje de materias del Ciclo Básico, el porcentaje de materias del Ciclo Avanzado, el porcentaje de las Complementarias y el porcentaje del Ciclo Introductorio.
    \1 Se necesita que puedan calcularse el porcentaje por área, que en el caso de la Licenciatura en Informática, estas áreas son: 
        \2 Programación.
        \2 Sistemas Informáticos.
        \2 Procesos Informáticos.
        \2 Desarrollo de Software.
        \2 Teoría de la Computación.
    \1 Se necesita cumplir con los pedidos específicos de CONEAU:
        \2 Postulantes: cantidad de alumnos que se anotaron.
        \2 Ingresantes: cantidad de alumnos que cursaron al menos una materia del Ciclo Introductorio.
        \2 Alumnos: Los que no son ingresantes. Alumnos activos >> Buscar definicion.
        \2 Tabla de ingresantes por cohorte.
        \2 Tabla de cursantes por cohorte.
        \2 Tabla de graduados por cohorte.
        \2 Tasas de crecimiento anual y por cuatrimestre de las carreras, cursantes y solo ingresantes.
    \1 Se necesita, con respecto a una materia determinada, conocer:
        \2 Cantidad de aprobados.
        \2 Cantidad de ausentes.
        \2 Cantidad de desaprobados.
    \1 Estudiantes con una determinada materia aprobada, que distinga en: materia aprobada por equivalencia, examen libre, pendiente, acta de cursada comun.
    \1 Poder ver de los inscriptos a una materia cuáles son recursantes, y cuántas veces.
    \1 Análisis retrospectivos y riesgos de deserción en base al historial general de la carrera.
    \1 Identificar cuáles son las materias que “traban” a los estudiantes en las carreras. 
    \1 Predicción de Inscriptos.
    \1 Análisis de cumplimientos de prerrequisitos según plan de estudios.
    
    
        
\end{outline}

\section[Conclusión]{Conclusión}

Al implementar esta solución, los datos estarán disponibles para ser consultados por las carreras. Además, existe un potencial para que esta solucion pueda ser extendida en el futuro y que esto implique una mejor experiencia también para los alumnos.
Llevar a cabo este proyecto implicará una mejora en cuanto a decisiones que se pueden tomar con respecto a los alumnos en pos de una mejor calidad educativa. Ayudará a determinar qué alumnos estan teniendo problemas, qué materias están trabando a los alumnos, cuáles son los alumnos que corren riego de desersión, cuáles están cerca de recibirse, etc.




		\chapter{Análisis de tecnologías}
\label{sec:analisis_tecnologias}

Para el desarrollo de la aplicación de gestión académica se analizaron diferentes tecnologías cuyas características mencionamos a continuación.

\section[Python]{Python}

Python es un lenguaje de programación de alto nivel, de propósito general e interpretado. Es un lenguaje open-source y por ende, el código fuente se puede obtener bajo la licencia GNU General Public License (GPL). Python permite programar de forma fácil y clara, tanto como para pequeña y gran escala proporcionando de varias construcciones. Su filosofia de diseño se enfatiza en la legibilidad, usando espacios en blanco. Más importante, Python incorpora los paradigmas de programación orientada a objetos, imperativo, funcional y procedural con una biblioteca estándar muy enriquecida.

\subsection[Es interpretado]{Es interpretado}

Python es procesado en tiempo de ejecución por el intérprete, no requiere compilación el programa entero antes de ser ejecutado. El intérprete directamente ejecuta el programa, línea por línea traduciendo cada declaración en una secuencia de subrutinas y luego a código máquina. Esto hace que sea más flexible que otros lenguajes de programación proporcionando un programa ejecutable más pequeño.

\subsection[Es orientado a objetos]{Es orientado a objetos}

Python es un lenguaje de programación multi paradigma. Por esto, soporta el paradigma orientado a objetos, las reglas y técnicas de programación que encapsulan código dentro de los objetos. Además, todo el código escrito en el fuente de Python está en forma de objetos y clases.

\subsection[Es portable]{Es portable}

Python tiene la capacidad de correr en una gran cantidad de plataformas de hardware con la misma interfáz. Corre perfectamente en todos los sistemas operativos como Windows, Linux, UNIX, Amigo, MacOs, etc.

\subsection[Es extensible]{Es extensible}

Python es tán versátil y flexible que permite a los programadores agregar módulos, de bajo o alto nivel, que ya existan o crear sus propios módulos para agregarlos al intérprete. Estos módulos y paquetes de herramientas permiten a los programadores desarrollar de una forma portable y multiplataforma, que les permite crear y personalizar sus programas, aplicaciones, o herramientas de software para ser más eficientes. 

\subsection[Es de scripting]{Es de scripting}
Python puede ser utilizado como un lenguaje de scripting, tanto Batch como Interactive. Además, sigue reglas de alcance simples, por lo tanto, se puede hacer un tipado dinámico flexible. Además, al ser un lenguaje de script, permite un fácil acceso a otros programas. Se puede compilar en código de bytes para construir aplicaciones grandes.

\subsection[Es web]{Es web}
Python ofrece una variedad de elecciones para el desarrollo de aplicaciones web, y esto se debe a su escalabilidad. La biblioteca estándar de Python incorpora muchos protocolos para el desarrollo web, como HTML, XML, JSON, procesamiento de emails. También provee las bases para FTP, IMAP, y otros protocolos de internet.

Siendo tan flexible, provee un fácil uso de la interfáz \textit{socket}, además de bibliotecas como Requests (un cliente HTTP), BeatifulSoup (parser HTML), Feedparser (parser RSS/Atom), entre otras. Además de las bibliotecas mencionadas, tiene varios \textit{frameworks} web como Django, Flask, Pyramid, etc. 


\section[Django]{Django}

Django es un framework de desarrollo web de código abierto, que respeta el patrón de diseño MVC (Modelo Vista Controlador). Fue diseñado como un framework del lado del servidor, que opera con bases de datos relacionales. Sin embargo, a lo largo de los años se fue adaptando a las necesidades y puede operar con bases de datos no relacionales (a través de paquetes de terceros).

El contenido en los proyectos de \textit{Django} trabaja en 3 grandes bloques: \textit{urls}, \textit{templates} y \textit{apps}. Se definen de forma separada y se conectan para cumplir con la entrega de contenido, que es parte de los principios de arquitectura poco acoplada que promueve Django.
Las \textit{URLs} definen puntos de entrada de acceso al contenido. Los \textit{templates} definen los puntos de salida que dan forma al contenido final. En el medio, las \textit{apps} sirven como \textit{middleware} entre las urls y los templates, agregando contenido desde una base de datos o la interacción de un usuario. 
Para servir código estático, sólo se necesitan crear y configurar \textit{urls} y \textit{templates}. Para contenido dinámico, se necesitan crear y configurar \textit{apps}.

\begin{figure}[h!]
  \centering
    \includegraphics[scale=0.9]{images/django.png}
  \captionof{figure}{Flujo de trabajo de Django}
  \label{fig:flujo_django}
\end{figure}

La Figura ~\ref{fig:flujo_django} muestra dos flujos de información. Uno se utiliza para enviar información estática, y el otro se utiliza para enviar información dinámica.

\section[Flask]{Flask}

Flask es un framework pequeño con respecto a los estándares de muchos frameworks web. Suficientemente pequeño para que sea llamado “micro-framework“\cite{Fsck}. Pero esto no significa que hace menos que otros frameworks, sino que fue diseñado para ser extensible desde los cimientos. Provee un núcleo sólido con los servicios básicos, mientras el resto puede ser provisto por extensiones. 

El hecho de que permita elegir y poner la extensión que se desee, hace que se termine usando un framework que no tiene funcionalidades de más y sirve para exactamente lo que se necesite.
Flask tiene tres dependencias principales: rutas, \textit{debugging} y \textit{WSGI} (\textit{Web Server Gateway Interface}, provisto por \textit{Werkzeug}); los \textit{templates} son provistos por \textit{Jinja2}, y la interacción por línea de comandos viene de \textit{Click}. Estas dependencias fueron escritas por Armin Ronacher, el autor de Flask.

\textit{Flask} no dispone de acceso nativo a bases de datos, validar formularios web, autenticación, u otras tareas de alto nivel. Sólo se pueden usar a través de extensiones. Como desarrollador, se puede decidir integrar la que se desee o escribir una propia. Esto va en contraste con grandes frameworks, ya que es complejo o imposible cambiar estas funcionalidades.

\section[Python en Data Science]{Python en Data Science}

La comunidad de la ciencia de datos está cambiando de R a Python, ya que provee a los científicos una gran cantidad de funcionalidades y les permite crear sus propias funcionalidades para realizar cálculos muy complejos. Además, les permite generar varios tipos de reportes de análisis, histogramas, grafos, y mucho más.
Existe una gran variedad de módulos de Data Science entre los que se destacan \textit{numpy} y \textit{Pandas}.

\subsection[Numpy]{Numpy}

NumPy, abreviatura de Numerical Python, es una parte fundamental para la computación numérica. Provee estructuras de datos, algoritmos y bibliotecas necesarias para la mayoría de las aplicaciones científicas que involucran datos numéricos en Python. NumPy contiene entre otras cosas:
\begin{outline}
    \1 ndarray: Un array multidimensional rápido y eficiente.
    \1 Funciones para realizar cálculos basados en elementos con matrices u operaciones matemáticas entre matrices.
    \1 Herramientas para leer y escribir datasets al disco.
    \1 Operaciones de álgebra lineal y generación de números aleatorios.
    \1 Una API en C muy madura que permite a las extensiones de Python acceder a las estructuras de datos de NumPy.
\end{outline}
Mas allá de las capacidades de procesamiento de matrices que NumPy le agrega a Python, una de sus principales usos para análisis de datos es como contenedor para que los datos sean pasados entre algoritmos y bibliotecas. Para datos numéricos, los arrays de NumPy son más eficientes para guardar y manipular los datos que cualquier otra biblioteca de Python.


\subsection[Pandas]{Pandas}

Pandas provee estructuras de datos de alto nivel y funciones diseñadas para trabajar con datos estructurados o tabulados de forma rápida, fácil y expresiva. Desde su salida en 2010 fue ayudando a Python a crear un entorno de análisis de datos poderoso y productivo. Los objetos principales de pandas son: 
\begin{outline}
    \1 DataFrame: una estructura de datos tabular, orientada a columnas, con etiquetas por columna y fila.
    \1 Series: un array etiquetado de una dimensión.
\end{outline}

Pandas mezcla la alta performance de las ideas computacionales sobre arrays de Numpy, con las capacidades flexibles de la manipulación de datos de las hojas de cálculos y las bases de datos relacionales (como SQL). Además, provee funcionalidades de indexación sofisticadas para que resulte fácil la remodelación, corte, agregación y selección de subconjunto de datos.

Pandas surgió en 2008 para cumplir con ciertos requerimientos que ninguna otra herramienta podía satisfacer:

\begin{outline}
    \1 Estructuras de datos con ejes etiquetados, y que soporte alineación de datos automática o explícita. Esto previene errores comunes resultantes de trabajar con datos provenientes de distintas fuentes.
    \1 Series de tiempo integradas.
    \1 Estructura de datos que pueda manejar series de tiempo y otros datos a la vez.
    \1 Operaciones aritméticas y reducciones que preserven los metadatos.
    \1 Manejo de datos faltantes de forma flexible.
    \1 Unión y otras operaciones relacionales encontradas en bases de datos.
\end{outline}

\section[React]{React}

React es una biblioteca de Javascript que tiene como propósito simplificar el desarrollo de interfaces visuales.
Fue desarrollado por Facebook y lanzado al mundo en 2013.
Su objetivo principal es facilitar el razonamiento sobre las interfaces y su estado en cualquier momento, dividiendo la UI en una colección de componentes.

\subsection[Componente]{Componente}

Los componentes permiten separar la interfaz de usuario en piezas independientes, reutilizables y pensar en cada pieza de forma aislada.
Conceptualmente, los componentes son como las funciones de JavaScript. Aceptan entradas arbitrarias (llamadas “props”) y devuelven a React elementos que describen lo que debe aparecer en la pantalla.
Los componentes pueden referirse a otros componentes en su salida. Esto nos permite utilizar la misma abstracción de componente para cualquier nivel de detalle. Un botón, un cuadro de diálogo, un formulario, una pantalla: en aplicaciones de React, todos son expresados comúnmente como componentes.


\subsection[DOM Virtual]{Dom Virtual}

DOM (Document Object Model) es un árbol que representa una página, empezando con la etiqueta <html>, bajando por cada hijo llamado nodo.
Se guarda en la memoria del navegador y se vincula directamente con lo que se ve en una página. El DOM tiene una API, con la cual se puede acceder a él, acceder a cada nodo, filtrarlos, modificarlos.
React mantiene una copia de la representación del DOM al que llama DOM virtual.

Cada vez que el DOM cambia, el navegador tiene que realizar dos operaciones intensivas: repintar (cambios visuales o de contenido en un elemento que no afectan el diseño y el posicionamiento en relación con otros elementos) y reflujo (recalcular el diseño de una parte de la página, o el diseño completo de la página).
React usa DOM Virtual para ayudar al navegador a usar menos recursos cuando se necesita que haya cambios en la página.

React sólo cambia el DOM cuando el estado de un componente cambia explícitamente.
Cuando hay un cambio, React actualiza el DOM Virtual relativo al componente que necesita cambiar.
La clave está en que sólo actualiza el DOM una vez, asi el \textit{repintado} y \textit{reflujo} que tiene que realizar el navegador se hace sólo una vez.


\section[Microservicios]{Microservicios}

Cuando se programan aplicaciones monolíticas (todas las funcionalidades en una sola aplicación), a medida que el código va creciendo cuando se agregan funcionalidades, se hace cada vez complejo saber dónde se tienen que hacer los cambios.
Las funcionalidades nuevas empiezan a quedar esparcidas por todo el código, haciendo que sea cada vez más difícil encontrar errores.


Los microservicios son servicios pequeños y autónomos que trabajan en conjunto para afrontan un acercamiento hacia la independencia de funcionalidades. Enfocándose en la obviedad donde el código solo sirve para alguna funcionalidad. Y al mantener este servicio enfocado dentro de sus límites, se evita que crezca demasiado.

\subsection[Heterogeneidad Tecnológica]{Heterogeneidad Tecnológica}

Con un sistema compuesto de múltiples servicios colaborando entre sí, podemos elegir usar diferentes tecnologías dentro de cada uno de éstos. Esto permite elegir la herramienta correcta para cada trabajo, en lugar de tener que elegir una opción que sirva un poco para todo, y puede terminar siendo perjudicial para el sistema.

Si una parte del sistema necesita mejorar su performance, podemos decidir cambiar de tecnología a una que resulte mejor para esa tarea en particular.

\begin{figure}[h!]
  \centering
    \includegraphics{images/heterogeneidad-tecnologica.png}
  \captionof{figure}{Microservicios: Homogeneidad Tecnológica}
  \label{fig:microht}
\end{figure}

\break

La Figura ~\ref{fig:microht} muestra tres servicios distintos, un núcleo que usa Django, un servicio de análisis de datos, con Flask y Pandas, y un servicio de visualización de datos con React.

\subsection[Independencia]{Independencia}

En un sistema monolítico, cuando hay una falla todo el sistema se corrompe. Aunque se pueda mitigar usando varias instancias del mismo sistema, esto no es conveniente ya que a medida que el sistema va creciendo, tener varias instancias de un sistema grande necesitaría una mayor capacidad de procesamiento.
Desde la perspectiva de microservicios, cuando uno de los servicios tenga una falla y esa falla no genere un problema en cascada, se puede aislar el problema y así el resto de los servicios puede seguir funcionando. 

\subsection[Escalamiento]{Escalamiento}

En un sistema monolítico, todo se escala junto. Si sólo se necesitara que un servicio determinado tenga más recursos, estaría dándole más recursos a todo el sistema.
En cambio, con microservicios, si en un determinado momento se necesita, por ejemplo, que el servicio de análisis de datos procese más peticiones se podría escalar sólo ese servicio teniendo dos instancias (o más) de esa parte del sistema, dejando los otros módulos corriendo con hardware menos poderoso, acorde a sus necesidades.

La Figura \ref{fig:microescala} muestra cómo podría ser escalado el sistema en caso de ser necesario. El Núcleo muestra 4 instancias, el análisis dos y la visualización una sola instancia, a modo de ejemplo.

\begin{figure}[h!]
  \centering
    \includegraphics[scale=0.7]{images/escalamiento.png}
  \captionof{figure}{Microservicios: escalamiento}
  \label{fig:microescala}
\end{figure}

\subsection[Facilitar deploy]{Facilitar deploy}

Hacer un pequeño cambio en un sistema monolítico requeriría \textit{deployar} toda la aplicación para hacer efectivo el cambio. Esto genera un gran impacto y un alto riesgo. 
Desde la perspectiva de los microservicios, el cambio que se hace sobre un servicio permite \textit{deployar} sólo ese módulo, sin que los otros sepan que hubo una modificación. Además, el código es desplegado más rápido. 
De haber un problema en la actualización, se puede aislar a un sólo servicio, permitiendo retroceder a una versión anterior.

\subsection[Reemplazabilidad]{Reemplazabilidad}

El costo de reemplazar un servicio por una mejor implementación es más fácil de manejar cuando hay independencia. 
Si en un futuro se quisiera, por ejemplo, cambiar el módulo de encuestas, sólo haría falta reemplazar ese pequeño sistema sin tener que modificar el resto. Ni siquiera deberían enterarse del cambio.

\section[REST]{Rest}

REpresentational State Transfer (Transferencia de estado representacional, su traducción) es un estilo arquitectural inspirado en la Web. 

Una parte importante es el concepto de \textit{recursos}. 
El servidor crea diferentes representaciones del recurso en un \textit{request} y la forma en que es enviado está desacoplado a cómo está guardado internamente. Cuando un cliente pide por un recurso, se le da una representación con la estructura de JSON.

A lo largo de los años, se desarrollaron muchos servicios basados en la arquitectura REST, que usa las funcionalidades provistas por la capa de aplicación del protocolo HTTP \cite{RestSoap} \cite{IETF}. Esto resultó en un incremento en el interés comparado al tradicional \textit{SOAP (Simple Object Access Protocol)}. Además, grandes compañías como Twitter o Amazon usan interfaces del tipo REST en sus servicios, lo cual se puede ver en la documentación de sus \textit{APIs (Application Programming Interface)}.
Mas allá de la tendencia, no hay estándares o guias de cómo desarrollar un servicio web RESTful. En lugar de esto, existen buenas prácticas.

\subsection[No versionado]{No versionado}

Versionar una API Web es una de las consideraciones más importantes a la hora de diseñar un servicio web, ya que la API representa el punto de acceso al servicio y oculta su implementación. Es por esto que una interfaz web nunca debería ser desplegada sin un identificador de versión \cite{WAPID}. Para versionar, existen diferentes acercamientos cómo incluirlo en la URI (Uniform Resource Identifier) del servicio web o usando el \textit{header HTTP} para seleccionar la versión apropiada \cite{WAPID}. Pero el servicio web basado en \textit{REST} no necesita ser versionado debido a \textit{hipermedia}. Es por esto que los servicios web \textit{RESTful} pueden ser comparados con los sitios web tradicionales, que pueden ser accedidos por todos los navegadores cuando se cambia el contenido del sitio. Así que no sería necesario agregar información del lado del cliente.


\subsection[Descripción de los recursos]{Descripción de los recursos}

La descripción de los recursos se relaciona con la usabilidad de los servicios web, ya que éstos son una representación del modelo. Para una correcta descripción, existen algunas buenas prácticas:
\begin{outline}
    \1 Deberían usarse sustantivos para los recursos \cite{WAPID}.
    \1 El nombre del recurso debería ser un nombre específico del dominio, para que la semántica pueda ser interpretada por cualquier usuario sin conocimientos adicionales \cite{WAPID}
    \1 Debería evitarse la mezcla del uso del plural y singular para garantizar coherencia \cite{WAPID}.
    \1 Debería usarse la convención de nombres de \textit{JavaScript}, ya que el tipo \textit{JavaScript Object Notation (JSON)} es el tipo de dato más usado para la comunicación entre el cliente y el servidor \cite{WAPID}.
\end{outline}
\subsection[Identificación de recursos]{Identificación de recursos}

Deberían usarse \textit{URIs} para la identificación única de recursos \cite{ASDNB}. Para esto existen algunas buenas prácticas:
\begin{outline}
    \1 Una URI debería ser autoexplicativa, de forma tal que no debería necesitar información adicional para su uso \cite{WAPID}.
    \1 No deberían existir verbos en la URI, ya que esto implica una orientación a métodos como SOAP \cite{WAPID}.
    \1 Un recurso debería ser represantado por dos URIs. La primera para representar el conjunto de estados de un recurso específico, y la segunda para representar un estado en particular de ese conjunto de estados \cite{WAPID}.
    \1 El identificador de un estado específico debería ser dificil de predecir y no referenciar objetos directamente, según \textit{OWASP (Open Web Application Security Project)}, si no existiera una capa de seguridad.
\end{outline}

\subsection[Manejo de errores]{Manejo de errores}

Los mensajes de error tienen que ser claros y entendibles, para que su causa pueda ser fácilmente identificada. Con este requerimiento se pueden reconocer algunas buenas prácticas:
\begin{outline}
    \1 La cantidad de códigos de estados de HTTP debería estar limitado para reducir el esfuerzo de buscar en la especificación \cite{WAPID}.
    \1 El uso de los códigos de estado de HTTP tienen que corresponderse con la especificación oficial de HTTP \cite{HTTP}.
    \1 Debería darse un mensaje de error detallado como una pista del error causado del lado del cliente \cite{WAPID}. Por este motivo, el mensaje de error debería tener los siguientes ingredientes:
        \2 Un mensaje para los desarrolladores, que describe la causa del error y algunas pistas sobre cómo resolver el problema.
        \2 Un mensaje que puede ser mostrado al usuario
        \2 Un código de error específico de la aplicación
        \2 Un link para más información sobre el problema.
\end{outline}
\subsection[Uso de parámetros]{Uso de parámetros}

Cada URI de un recurso puede ser extendida con parámetros para proveer información opcional al servicio. A continuación se detallan cuatro casos de uso:
\begin{outline}
    \1 Filtrado: para filtrar información de un recurso, puede usarse sus atributos o algún lenguaje de queries. La elección de una de estas variantes, depende de la necesidad del poder de expresión que se tenga para filtrar. 
    \1 Ordenamiento: para ordenar la información, se recomienda \cite{BPRA} una lista de atributos separados por coma con el parámetro \textit{sort} en la \textit{URI}, seguido por un signo de más (+) como prefijo para un orden ascendente, o un signo menos (-) para un orden descendente.
    \1 Selección: la selección de información en forma de atributos reduce el tamaño de transmisión sobre la red, respondiendo sólo con la información pedida. Para este propósito, se recomienda una lista de atributos separadas por coma.
    \1 Paginación: La paginación permite partir la información en varias páginas virtuales, mientras se referencian la página anterior, la próxima, la primera y la última. 
\end{outline}
\subsection[Interacción con los recursos]{Interacción con los recursos}

Usando REST como el estilo de arquitectura subyacente de un sistema, el cliente interactúa con las representaciones de un recurso en lugar de usarlo directamente. La interacción entre el cliente y el servidor está construido en la capa de aplicación del protocolo \textit{HTTP}, que ya provee cierta funcionalidad para la comunicación. Para la interacción con un recurso, podríamos identificar tres buenas prácticas diferentes:
\begin{outline}
    \1 El uso de los métodos HTTP deberían ser de acuerdo a las semánticas definidas por la especificación oficial de HTTP \cite{WAPID}. Así, el método GET de HTTP sólo debería ser usado para las operaciones sin efectos secundarios. Para una mejor visión general, la Tabla ~\ref{tab:tabla_http} muestra los métodos HTTP más usados y sus características. Estas características pueden ser usadas para asociar los métodos HTTP con el correcto uso de la creación, lectura, edición y borrado de recursos (CRUD) \cite{RVINOSKI}.
    \1 El soporte de la operación OPTIONS es recomendada si una gran cantidad de datos tienen que ser transmitidos, ya que permite al cliente pedir los métodos soportados de la representación actual antes de transmitir la información por un medio compartido. 
    \1 El soporte del GET condicional debería ser considerado durante el desarrollo de un servicio basado en HTTP, ya que previene al servidor de tranmitir datos ya enviados anteriormente. Sólo si hay modificaciones de la información solicitada desde la última solicitud, el servidor responde con la última representación  \cite{RVINOSKI}.
\end{outline}

\begin{table}[!htbp]
    \centering
    \makegapedcells
    \begin{tabular}{|c|c|c|}
    \hline
    Método & Seguro & Idempotente \\ \hline
    POST & No & No \\ \hline
    GET & Si & Si \\ \hline
    PUT & No & Si \\ \hline
    DELETE & No & Si \\ \hline
    
    \end{tabular}
    \caption{Características de los métodos HTTP más comúnes}
    \label{tab:tabla_http}
\end{table}

\subsection[REST y HTTP]{REST y HTTP}
HTTP define algunas funcionalidades que son muy útiles para REST. Por ejemplo los verbos (GET, POST, PUT, etc) ya tienen un buen entendimiento sobre cómo deberían funcionar con recursos. El estilo de arquitectura REST nos dice que los métodos deberían comportarse de la misma forma en todos los recursos, y la especificación HTTP define muchos de estos métodos. 
GET sirve para pedir recursos y POST para crear (aunque para el sistema seguimiento académico de no todos los recursos van a poder ser creados).
HTTP también trae un gran ecosistema de herramientas y tecnologías que facilitan la tarea. 


\section[JSON Web Tokens (JWT)]{JSON Web Tokens (JWT)}
\subsection[¿Qué es?]{¿Qué es?}

JSON Web Token es un estándar abierto (RFC 7519) que define una forma compacta de transmitir información entre partes de forma segura como un objeto \textit{JSON}. Esta información puede ser verificada y confiada porque está firmada digitalmente. JWT puede ser firmado usando una clave secreta (con el algoritmo \textit{HMAC}) o un par de claves pública/privada usando RSA o ECDSA.


\subsection[¿Cuándo usarlo?]{¿Cuándo usarlo?}

Estos son algunos escenarios donde JWT es útil:
\begin{outline}
    \1 Autorización: Este es el escenario mas común. Una vez que el usuario está \textit{logueado}, cada \textit{request} debe incluir el JWT, permitiendo al usuario acceder a rutas, servicios y recursos que le son permitidos con ese token. Single Sign On es una funcionalidad que usa mucho JWT, por su habilidad de poder ser usado a través de diferentes dominios.
    \1 Intercambio de información: JSON Web Tokens son una forma segura de transmitir información entre partes. Como están firmados, se puede estar seguro que el emisor es quien dice qué es. Además, como la firma es calculada usando el \textit{header} y el \textit{payload}, se puede verificar que el contenido no fue modificado.
\end{outline}
\subsection[Estructura]{Estructura}


En su forma compacta, los JSON Web Tokens consisten de tres partes separadas por puntos (.), que son:
\begin{outline}
    \2 Header
    \2 Payload
    \2 Firma
\end{outline}

\subsection[Header]{Header}

El header consiste de dos partes: el tipo de token, que es JWT, y el algoritmo que se usa para firmar, como HMAC SHA256.

\begin{minted}[frame=single, framesep=3mm, linenos=true, xleftmargin=21pt, tabsize=4]{js}
{
"typ": "JWT",
"alg": "HS256"
}
\end{minted}
Luego, este JSON es encodeado con Base64Url para formar la primer parte del JWT.

\subsection[Payload]{Payload}

La segunda parte del token es el \textit{payload} que contiene los pedidos. Un pedido es una declaración sobre una entidad (por lo general, un usuario) e información adicional. Hay tres tipos de pedidos: registrados, públicados y privados.

\begin{outline}
    \1 Registrados: Estos son un conjunto de pedidos predefinidos, que no son obligatorios pero recomendados para proveer información útil. Algunos de estos son: iss (issuer), exp (fecha de expiración del token), sub (asunto), aud (audiencia), y otros.
    \1 Públicos: Esta parte es definida por quien use el JWT. Pero para evitar colisiones, deberían estar definidos en el registro \textit{IANA} o estar definidos como una \textit{URI} que contiene un nombre resistente a colisión.
    \1 Privados: Estos son campos personalizados, con la finalidad de compartir información entre partes.
\end{outline}

Un ejemplo de payload puede ser: 

\begin{minted}[frame=single, framesep=3mm, linenos=true, xleftmargin=21pt, tabsize=4]{js}
{
  "token_type": "refresh",
  "exp": 1582059853,
  "jti": "e9b85778f3f44decba69a611e4e1c700",
  "user_id": 1
}
\end{minted}

El payload luego es encodeado con Base64Url para formar la segunda parte del JWT.

\subsection[Firma]{Firma}

Para crear la parte de la firma, se usa el \textit{header encodeado}, el \textit{payload encodeado}, una clave secreta, y el algoritmo especificado en el \textit{header}.
Por ejemplo, usando el algoritmo HMAC SHA256, la firma se crea de la siguiente forma: 

\begin{lstlisting}[language=Python]
HMACSHA256(
    base64UrlEncode(header) + “.“ +
    base64UrlEncode(payload),
    clave-secreta
)
\end{lstlisting}
La firma es usada para verificar que el mensaje no fue cambiado en el camino, y también para verificar que el emisor es quien dice ser.

\subsection[Poniendo todo junto]{Poniendo todo junto}

El resultado van a ser tres strings en Base64Url, separados por puntos que pueden ser fácilmente pasados por HTTP, siendo muy compactos comparado con estándares XML como SAML.

Lo siguiente muestra un JWT formado con las tres partes provistas:


eyJhbGciOiJIUzI1NiIsInR5cCI6IkpXVCJ9.\break eyJ0b2tlbl90eXBlIjoicmVmcmVzaCIsImV4cCI6MTU4MjA1OTg1MywianRpIj\break oiZTliODU3NzhmM2Y0NGRlY2JhNjlhNjExZTRlMWM3MDAiLCJ1c2VyX2lkIj\break oxLCJjYXJyZXJhcyI6WyJXIl0sInVzZXJuYW1lIjoiYWRtaW4ifQ.\break KryBoqAQpHNliUUw3-xLJE2u4M6\_tBByJQtf7IFYQis



\subsection[¿Cómo funciona?]{¿Cómo funciona?}

En autenticación, cuando el usuario se loguea usando sus credenciales, se le provee un JSON Web Token. Como es una credencial, se debe tener cuidado con los problemas de seguridad que se puedan tener. En general, no se deben mantener tokens más tiempo del requerido. Tampoco se tiene que guardar datos sensibles de sesiones en el almacenamiento del navegador.
Cuando un usuario quiere ingresar a una ruta o un recurso protegido, tiene que mandar el JWT en el header de autorización, usando el esquema “Bearer“. El contenido del header tiene que verse como lo siguiente:
\begin{lstlisting}[language=Python]
Authorization: Bearer <token>
\end{lstlisting}

Este puede ser un mecanismo de autorización sin estado. Las rutas protegidas del servidor chequean que el JWT sea válido. Si lo es, el usuario puede ingresar a esas rutas protegidas. Si el JWT tiene los datos necesarios, se reducen las queries a la base de datos, aunque no siempre sea el caso.
Si el token es enviado en el header de autorización, Cross-Origin Resource Sharing (CORS) no va a ser un problema ya que no usa cookies.


\subsection[Problemas que resuelve]{Problemas que resuelve}

A pesar de que el principal propósito de JWTs es transferir demandas entre dos partes, el aspecto más importante es el de estandarizar una estructura de datos de forma simple y encriptada. 
Los principales problemas que resuelve son:
\begin{outline}
\2 Autenticación
\2 Autorización
\2 Identidad Federada
\2 Sesiones del lado del cliente
\end{outline}

\subsection[Client-side/Stateless Sessions]{Client-side/Stateless Sessions}

Las llamadas sesiones sin estado (stateless sessions) son en realidad datos del lado del cliente (\textit{client-side}). El aspecto fundamental de esta aplicación se basa en el uso de firmas y posiblemente encriptación para proteger el contenido de la sesión. 

Como los datos del lado del cliente pueden ser manipulados con facilidad se tiene que tener mucho cuidado desde el \textit{backend}.

JWTs, en virtud de JWS y JWE, puede proveer distintos tipos de firmas y encriptación. Las firmas son útiles para validar los datos contra posibles manipulaciones. La encriptación es útil para proteger los datos de ser leídos por terceros.

\subsection[¿Es útil tener una sesión del lado del cliente?]{¿Es útil tener una sesión del lado del cliente?}

Existen pros y contras a cualquier decisión, y las sesiones \textit{client-side} no son la excepción. Algunas aplicaciones pueden requerir sesiones muy grandes. Enviando este estado ida y vuelta hacia el \textit{backend} por cada \textit{request} (o grupo de \textit{requests}) puede vencer rápidamente los beneficios que trae \textit{JWT}. Es necesario un buen balance entre los datos del lado del cliente y las búsquedas a la base de datos del \textit{backend}.

\subsection[Tokens de acceso y refresh]{Tokens de acceso y refresh}

Los \textit{tokens} de acceso (\textit{access}) y \textit{refresh} son dos tipos de \textit{tokens} que ayudan en el contexto de autenticación y autorización.

Los tokens de acceso son \textit{tokens} que dan a aquel que lo tenga, el acceso a recursos protegidos. Estos tokens son de corta vida y tienen una fecha de expiración como dato. Tienen, además, otra información que puede ser de ayuda para identificar al cliente. Esta información adicional está definida en la implementación.

Por el contrario, los \textit{tokens} de \textit{refresh} le da permiso a los clientes que pidan un nuevo acceso. Por ejemplo, luego de que un token de acceso expiró, un cliente puede hacer un pedido de un nuevo acceso al servidor de autorización. Para que ésto pueda suceder, es requerido un token de \textit{refresh}.
A diferencia de los \textit{tokens} de acceso, el tiempo de vida del token de \textit{refresh} suele ser largo.

La principal diferencia entre el token de acceso y el de \textit{refresh}, está en la posibilidad de hacer los tokens de acceso fáciles de validar. Un token de acceso que tiene una firma, no hace falta que sea validado por un servidor de autorización.
Los \textit{tokens} de \textit{refresh}, por el contrario, necesita que se acceda al servidor de autorizaciones. Manteniendo la validación separada de las \textit{queries} al servidor de autorización, es posible obtener una mejor latencia.
Para asegurar que la pérdida o robo de tokens no sea determinante, se debería poner un tiempo de vida corto.
Los tokens de refresh, al ser de vida prolongada, tienen que ser protegidos contra estas incidencias. 
Una opción es agregarlo a una lista negra y cuando alguien pida un access token, éste es expirado automáticamente.

\section[Docker]{Docker}

\subsection[Introducción]{Introducción}

Docker es una plataforma abierta para desarrollar, transportar, y ejecutar aplicaciones de forma rápida. Docker permite que las aplicaciones corran de forma separada de la infraestructura del \textit{host}. Permite también enviar código, testear, desplegar rápido y acortar el ciclo entre escribir el código y correrlo. Docker logra esto combinando una plataforma de virtualización liviana con herramientas que permiten manejar y desplegar aplicaciones \cite{Dj}.

Docker usa funcionalidades del kernel de Linux como \textit{cgroups}, que limita y aisla el uso de recursos, y \textit{namespaces}, que permite a los contenedores independientes correr en una única instancia de Linux evitando la sobrecarga de iniciar máquinas virtuales.
Además, provee una forma para correr aplicaciones aisladas en un contenedor de forma segura, y que estos contenedores puedan correr de forma simultánea en un mismo \textit{host} compartiendo el mismo kernel. Aunque cada contenedor puede usar una cantidad de recursos definidos por el \textit{host}.

A diferencia de las máquinas virtuales, no se requiere un sistema operativo separado, sino que depende de las funcionalidades del \textit{kernel} y el uso aislado de recursos (CPU, memoria, I/O, red, etc).

\subsection[imágenes y contenedores]{imágenes y contenedores}

Un contenedor (container) es una versión de un sistema operativo Linux, solo con los componentes más básicos. Una imagen es un software que se carga dentro del contenedor al momento de ejecutar el comando run

\begin{lstlisting}[language=bash]
    docker run hello-world
\end{lstlisting}


El comando run recibe como parámetro requerido el nombre de la imagen que se desea cargar en un contenedor, en este caso, hello-world.
Al correr dicho comando, Docker ejecuta las siguientes acciones:

- Comprobar que exista en el sistema una imagen con el nombre hello-world.
- En caso que no exista, se descarga desde el repositorio de imágenes configurado (por defecto es Docker Hub).
- Cargar la imagen en el contenedor y ejecutarla.

Por otro lado, una imagen de Docker puede ejecutar desde un simple comando hasta cargar un complejo sistema de base de datos.
Para construir una imagen, es necesario crear un archivo llamado Dockerfile.

\begin{lstlisting}
    FROM ubuntu:16.04
    RUN apt-get -y update
    CMD[“echo Hola“]
\end{lstlisting}

El dockerfile anterior busca una imagen de Ubuntu con el tag 16.04. Luego ejecutará un comando para actualizar los paquetes del sistema operativo y finalmente mostrará el mensaje Hola.
El comando para construir una imagen de Docker es:

\begin{lstlisting}[language=bash]
    docker build -t nombre-imagen
\end{lstlisting}

Se ejecutará el comando build para construir la imagen. El argumento -t indica que se pondrá la etiqueta nombre-imagen a la imagen. El punto final indica el directorio de contexto de la imagen. En este caso, el contexto será el directorio donde se encuentra el Dockerfile.
Luego se carga la imagen en un contenedor con el comando:

\begin{lstlisting}[language=bash]
    docker run nombre-imagen
\end{lstlisting}


\subsection[Contenedor]{Contenedor}

Los contenedores y las máquinas virtuales tienen un aislamiento y asignación de recursos similar aunque funcionan diferente ya que los contenedores virtualizan el sistema operativo en lugar del hardware.

\begin{figure}[h!]
  \centering
    \includegraphics[scale=0.7]{images/containers.png}
  \captionof{figure}{Estructura de contenedores}
  \label{fig:contvm}
\end{figure}

\break

Los contenedores son una abstracción de la capa de aplicación que empaqueta el código y las dependencias juntos. Múltiples contenedores pueden correr en la misma máquina compartiendo el kernel del sistema operativo junto con otros contenedores, todos corriendo de forma aislada (Figura \ref{fig:contvm}). Además, los contenedores ocupan menos espacio que las VMs.

\begin{figure}[h!]
  \centering
    \includegraphics[scale=0.7]{images/vms.png}
  \captionof{figure}{Estructura de Máquinas virtuales}
  \label{fig:vm}
\end{figure}

\break

Las máquinas virtuales (VMs) son una abstracción del hardware físico, transformando un servidor en múltiples servidores. El \textit{hypervisor} permite que múltiples VMs corran en una misma máquina (como muestra la Figura \ref{fig:vm}). Cada VM incluye una copia de un sistema operativo, la aplicación, los binarios y bibliotecas, ocupando decenas de GBs. 

\subsection[Docker Compose]{Docker Compose}

Docker Compose es una herramienta que permite correr un sistema formado por múltiples contenedores. Para ello, se debe crear un archivo .yml en el que se definan los servicios con los que va a contar la aplicación. Cada servicio estará formado por un contenedor corriendo una imagen de Docker. 
Para cada servicio pueden definirse nombres, puertos expuestos, conexiones de red, etc. Luego, con los siguientes comandos se puede operar con el sistema.

\section[Nginx]{Nginx}

Nginx es uno de los web servers más populares. Sirve tráfico HTTP y HTTPS, proxys a Python, NodeJS, corre software como balanceador de carga, http cache, SSL, etc.
Nginx es una pieza de software extremadamente modular. Muchas de las funcionalidades que trae por defecto, son en realidad módulos que se pueden activa y desactivar en cualquier momento. Una de las ventajas es que se puede decidir qué módulos se quiere utilizar, y cuales no.
Una de las funcionalidades más importantes para el desarrollo de microservicios es el proxy reverso.

\subsection[Forward proxy vs reverse proxy]{Forward proxy vs reverse proxy}

Una de las razones más populares para usar nginx es para que sirva aplicaciones dinámicas escritas en Python, NodeJS, Ruby, y muchas otras.
A diferencia de Apache, nginx no tiene la habilidad de embeber un intérprete al webserver, como hace Apache con mod\_php. En su lugar, nginx toma un enfoque mucho más liviano. Es simplemente un webserver y su tarea principal es correr una aplicación web delegandolo a un server y proxies separados.

\subsection[Forward Proxy]{Forward Proxy}

Se le llama \textit{forward proxy} a la configuración por defecto que se encuentra en todas las conexiones salientes de internet. La Figura \ref{fig:forwardproxy} muestra la estructura de funcionamiento.

\begin{figure}[h!]
  \centering
    \includegraphics[scale=0.7]{images/forward-proxy.png}
  \captionof{figure}{Funcionamiento de Forward Proxy}
  \label{fig:forwardproxy}
\end{figure}

Las conexiones salientes de una computadora son capturadas por el \textit{forward proxy} y reenviadas a Internet. Para Internet, todas las computadoras aparecen como si vinieran del mismo lugar- el \textit{forward proxy}.


\subsection[Reverse Proxy]{Reverse Proxy}

Se le llama \textit{reverse proxy} a lo opuesto de \textit{forward proxy}, y es una configuración muy común para servir aplicaciones web dinámicas y para balance de carga.

\begin{figure}[h!]
  \centering
    \includegraphics[scale=0.7]{images/reverse-proxy.png}
  \captionof{figure}{Funcionamiento de Reverse Proxy}
  \label{fig:reverseproxy}
\end{figure}

En este ejemplo, el \textit{reverse proxy} hace de multiplexor para muchas conexiones de Internet, a una aplicación dinámica. El \textit{reverse proxy} mira el request, y lo reenvia a la aplicación.

\subsection[Balance de carga]{Balance de carga}

Una de las formas más comunes de escalar una aplicación web es usar \textit{balance de carga}. La idea detrás de esto, es distribuir la carga, o tráfico web, a través de diferentes \textit{application servers}.
El funcionamiento normal de una sola aplicación con nginx es que los visitantes ingresan el nombre de dominio y son ruteados directamente al servidor donde nginx está sirviendo la única aplicación. El problema es que cuando se tienen miles de conexiones visitando la aplicación, es probable que un solo servidor no pueda manejarlo. 
Ahí es cuando juega un papel importante el balanceador de carga, que recibe todo el tráfico ingresante y lo distribuye a través de un conjunto de servidores de aplicación.
Existen distintos tipos de balanceos de carga: por software y por hardware. Nginx tiene la habilidad de correr como un balanceador de carga de software, usando el módulo http\_proxy.

El balance de carga es en definitiva un \textit{reverse proxy}, con tres diferencias fundamentales: 
\begin{outline}
\1 Los balanceadores de carga reenvían el tráfico a través de varios \textit{backends}, mientras que un reverse proxy tradicional lo hace hacia uno solo.
\1 Los balanceadores de carga usualmente operan en la capa 7 (HTTP) o en la capa 4 (TCP) del modelo OSI, cuando típicamente sólo operaríamos en la capa 7 usando un reverse proxy de aplicaciones modernas.
\1 La escala es crítica para los balanceadores de carga: en sitios ocupados, pueden ver 20 veces la cantidad de tráfico que recibe un servidor de aplicación. Por ejemplo, un servidor de aplicaciones potente solo necesitará manejar entre 100 y 200 solicitudes por segundo, mientras que un balanceador de carga puede recibir más de 10,000 solicitudes por segundo.
\end{outline}


\section[Conclusiones]{Conclusiones}

Luego de investigar las diferentes tecnologías, se decidió que la mejor solución era desarrollar tres aplicaciones diferentes. Por un lado, un núcleo que se encargase de manejar los datos y servirlos mediante una API Rest. Esto sería beneficioso para el seguimiento académico pero también para que estos datos puedan ser consultados por aplicaciones futuras. 

Luego, para el análisis y visualización de los datos se decidió separarlo en dos aplicaciones diferentes. La primera, desarrollada con Flask, que conjuntamente con Pandas y Numpy se encargará de analizar y servir los datos analizados. La segunda, desarrollada con React, tendrá la tarea de mostrar los datos analizados.

La implementación de estas tres aplicaciones se detalla en el capítulo siguiente.
		\chapter{Desarrollo de la Solución}
\label{sec:desarrollo}

\section[Proceso de Desarrollo]{Proceso de Desarrollo}

\subsection{Versionado de código}

\subsection{Integración Continua}

\subsection{Tests}

\section[Núcleo]{Núcleo}

El núcleo se encarga de administrar los datos no sensibles de los alumnos y sus materias cursadas, las inscripciones, planes de estudio con sus créditos, recorrido obligatorio y recomendado de inscripciones, entre otros.
El núcleo tiene la capacidad de servir dichos datos para que sean consumidos por los usuarios que tengan los permisos correspondientes.
Los usuarios tienen permisos asignados que corresponden con la carrera a la que pertenecen, que les permiten consultar datos de dicha carrera. 






\begin{figure}[h!]
  \centering
    \includegraphics[scale=0.5]{images/nucleo/nucleo-fondoblanco.png}
  \captionof{figure}{Logo del Núcleo}
  \label{fig:django}
\end{figure}

\subsection{Tecnologías}

El Núcleo fue desarrollado con Python 3.6, usando el framework web Django en su versión 3.0.
Django incluye un administrador (django-admin) que genera automáticamente los listados, la creación, la edición y el borrado de los modelos desarrollados. Además, trae un sistema de autenticación de usuarios que facilita la tarea de autorizar usuarios a través de permisos.
Se eligió django-rest-framework para crear una API REST para que otros servicios puedan consultar o crear datos.
Usando django-rest-framework-simplejwt, se implementó JWT. Es decir, todo servicio que quiera acceder a los datos tendrá que pedir un token. Si éste está habilitado para consumir datos, se le proveerá de dicho token.

\subsection{Administración de los datos}

Los usuarios que tengan permiso para ingresar al Administrador, pueden hacerlo mediante una pantalla de Login como muestra la figura ~\ref{fig:nucleo-login}.
Una vez ingresados, pueden ver la pantalla principal del administrador, como muestra la figura ~\ref{fig:nucleo-home}, donde tienen diferentes menús para utilizar los datos. Tienen acceso a diferentes listados (figura ~\ref{fig:nucleo-listado}), creación, edición y borrado (figura ~\ref{fig:nucleo-edicion}).
Los usuarios tienen también la posibilidad de importar planillas de datos (figura ~\ref{fig:nucleo-importador}).

\subsubsection{Usuarios y permisos}

Para resolver la autenticación y autorización en el administrador de contenidos, se usó el sistema de autenticación que viene embebido en Django.
Este sistema provee una manera de asignar permisos a usuarios específicos o grupos de usuarios. Estos permisos se dividen en:
\begin{outline}
    \1 El acceso a ver objetos está limitado a los usuarios que tengan los permisos \textit{view} o \textit{change}.
    \1 El acceso a ver los formularios para agregar elementos, está limitado a los usuarios que tengan el permiso \textit{add} para ese tipo de objetos.
    \1 El acceso a ver los listados, ver los formularios de edición y la posibilidad de editar, están limitados a los usuarios que tengan el permiso \textit{change} para ese objeto en particular.
    \1 El acceso a eliminar un objeto está limitado a los usuarios que tengan el permiso \textit{delete}.
\end{outline}

\subsubsection{Grupos de usuarios}

Django provee una forma de categorizar usuarios a los que se les puede aplicar permisos llamado \textit{grupo}. Un usuario puede pertenecer a muchos grupos.
Un usuario que pertenezca a un grupo, automáticamente tiene los permisos asignados a dicho grupo.
Otra particularidad de esto, además de los permisos, es que pueden servir para extender las funcionalidades. Por ejemplo, si quisiera que un grupo fuese "Usuarios de LIDS", podría asignarle a ese grupo la carrera "LIDS" para que sólo pudieran ver datos relacionados a su carrera y no a otras.


\subsection{Importadores}

En el caso que los usuarios tengan la necesidad de importar los datos a través de planillas, se realizaron diferentes importadores para esta tarea (~\ref{fig:nucleo-importador}).
Estos importadores son:
\begin{outline}
\2 Carreras
\2 Planes de estudio con sus materias
\2 Prerrequisitos obligatorios y recomendados de materias
\2 Alumnos con sus datos personales
\2 Materias cursadas por alumnos
\2 Inscripciones a materias
\end{outline}

\subsection{API}

Se diseñó una API para consultar datos a través de distintas URIs (tabla ~\ref{tab:tabla_api}), para las cuales se necesita de un token (tabla ~\ref{tab:tabla_token}).
Dicho token es de la forma:

\textit{eyJ0eXAiOiJKV1QiLCJhbGciOiJIUzI1NiJ9.} \break 
\textit{eyJ0b2tlbl90eXBlIjoiYWNjZXNzIiwiZXhwIjoxNTg5Mjk2NjAyLCJ}\break 
\textit{qdGkiOiJhMTAzMmI2YzdiN2Y0ZjlkODc5NzI0NGViZTQxYTk5YSIsInV}\break 
\textit{zZXJfaWQiOjEsImNhcnJlcmFzIjpbIlciXSwiY2FycmVyYXNfbGFiZWwi} \break \textit{OltbIlciLCJMaWNlbmNpYXR1cmEgZW4gRGVzYXJyb2xsbyBkZSBTb2Z0d2}\break 
\textit{FyZSJdXSwidXNlcm5hbWUiOiJhZG1pbiJ9}.\break 
\textit{xWWi-sDFQ6I-CK0xC7tmkTw1mXRmMhFFse6\_qnKBiaE}

\break
El \textit{Header} del token tiene la siguiente información:
\begin{lstlisting}[language=json,firstnumber=1]
{
"typ": "JWT",
"alg": "HS256"
}
\end{lstlisting}
\break
Para el caso del \textit{Payload}, se tiene la siguiente información indicando precisamente qué carreras puede consultar el usuario:
\begin{lstlisting}[language=json,firstnumber=1]
{
  "token_type": "access",
  "exp": 1589296602,
  "jti": "a1032b6c7b7f4f9d8797244ebe41a99a",
  "user_id": 1,
  "carreras": [
    "W"
  ],
  "username": "admin"
}
\end{lstlisting}

\subsubsection{Ejemplo de pedido}
Si quisiera saber qué materias hay dentro de una carrera en un plan determinado, deberia hacer de la siguiente manera:

\begin{lstlisting}[language=bash]
GET 'https://url.del.nucleo/api/carreras/W/planes/2015/'
--header Authorization: Bearer <token>
\end{lstlisting}

El resultado tendrá la siguiente forma:

\begin{lstlisting}[language=json,firstnumber=1]
[{
        "id": 54,
        "materia": "Taller de Trabajo Intelectual",
        "plan": 2015,
        "nucleo": "",
        "creditos": 4,
        "area": "Taller",
        "codigo": "00751"
    },
    ...
]
\end{lstlisting}

\begin{figure}[h!]
  \centering
    \includegraphics[scale=0.8]{images/nucleo/jwt.png}
  \captionof{figure}{Funcionamiento del pedido de token y posteriores requests}
  \label{fig:nucleo-jwt}
\end{figure}


\subsection{Deploy}

Docker, nginx, base de datos

\section[Análisis de Datos]{Análisis de Datos}

\subsection{Tecnologías}

\subsection{API REST}

\subsection{}

\section[Visualización de los datos analizados]{Visualización de los datos analizados}

\subsection{Tecnologías}


		\chapter{Pruebas realizadas}
\label{sec:implementacion}

\section[Pruebas y resultados]{Pruebas y resultados}

\subsection[Resultados esperados]{Resultados esperados}

Se espera con este desarrollo tener un sistema capaz de brindarle datos a todos los interesados. Para que esto se cumpla, la implementación tiene que ser capaz de manejar todos los pedidos que recibe y responder en consecuencia. 

Si bien la problemática inicial implica que sólo se consultarán datos para analizarlos y ser visualizados por los directores de las carreras de la Universidad Nacional de Quilmes, no hay que descartar que pueda ser extendido dado su potencial.

Para garantizar que todo funcione de manera correcta se realizaron diferentes pruebas con escenarios cambiantes.
Para todas las pruebas se eligieron 5 URIs y se realizaron sobre una base de datos con 1.424 alumnos y 13.234 materias cursadas por estos alumnos.
La arquitectura utilizada para las pruebas se asemeja a la que dispondrá a la hora de la implementación.

Se utilizó la herramienta jMeter para realizar los requests y evaluar los tiempos de respuesta y errores, entre otras métricas.

Para medir los tiempos de respuesta, se usará APDEX.

APDEX (Índice de Performance de la Aplicación, por sus siglas en inglés) es un estandard abierto para medir la performance de una a aplicación de software. Su propósito es convertir las mediciones en información sobre la satisfacción del usuario, al especificar una forma uniforme de analizar e informar sobre el grado en que el rendimiento medido cumple con las expectativas del usuario.
Para realizar una medición APDEX, es necesario establecer un tiempo \emph{satisfactorio} y un tiempo \emph{tolerable}.
Una vez establecidos estos valores y tendo los resultados de la prueba, se calcula el índice:

\begin{align*}
  Apdex = \frac{Satisfactorios + \frac{Tolerables}{2}}{Total}\\
\end{align*}

\break
\subsection{Pruebas con 100 usuarios simultáneos}
En primer lugar, se realizó una prueba en la cual 100 usuarios simultáneos se encuentran haciendo pedidos al núcleo, el cual está desplegado en una sola instancia de docker.
Esta prueba duró 5 minutos y los resultados se muestran a continuación.


\subsubsection{Resultados generales de la prueba}


Los resultados generales, como muestra la tabla ~\ref{tab:100u_5m_gen}, indican que durante los 5 minutos que duró la prueba se realizaron 16.997 requests en total, y el núcleo pudo responder sin ningún error.

El tiempo promedio de respuesta fue de 1043.79ms.

\begin{table}[!htbp]
    \centering
    \makegapedcells
    \begin{tabular}{|c|c|c|c|}
    \hline
    Nombre del Pedido & Cantidad & Errores (\%) & Tiempo Promedio (ms) \\ \hline
    Total & 16977 & 0.00\% & 1043.79\\ \hline
    Carrera | graduados & 2834 & 0.00\% & 1038.98\\ \hline
    Carrera | ingresantes & 2871 & 0.00\% & 1057.51\\ \hline
    Carrera | postulantes & 2850 & 0.00\% & 1046.61\\ \hline
    Carreras & 2793 & 0.00\% & 1851.00\\ \hline
    Plan | Materias necesarias & 5629 & 0.00\% & 1036.10\\ \hline
    
    \end{tabular}
    \caption{Pruebas realizadas con 100 usuarios simultáneos durante 5 minutos}
    \label{tab:100u_5m_gen}
\end{table}


\subsubsection{Resultados sobre el uso de recursos durante la prueba}

Para evaluar el consumo de recursos, se utilizó el comando \textit{docker stats} que realiza una transmisión en vivo de los recursos que van consumiendo los contenedores.
En promedio utilizó un 68\% del CPU y 55MB de memoria RAM (tabla ~\ref{tab:100u_5m_rec}).


\begin{table}[!htbp]
    \centering
    \makegapedcells
    \begin{tabular}{|c|c|c|}
    \hline
    Contenedor & CPU & Memoria (MB)\\ \hline
    Núcleo & 68\% & 55 \\ \hline
    \end{tabular}
    \caption{Recursos en promedio consumidos por el contenedor durante la prueba}
    \label{tab:100u_5m_rec}
\end{table}


\subsubsection{APDEX de la prueba}

Para determinar el APDEX de la prueba, se estableció que la \textit{tolerancia} es de 2 segundos, y la \textit{frustración} es de 3 segundos. 

Como se explicó anteriormente, esta prueba tiene en cuenta el tiempo que tardan todos los requests para luego determinar este valor.

Para esta prueba, el APDEX fue de 0.995, lo cual es casi perfecto (el valor máximo es 1) dentro de los parámetros elegidos.

La tabla ~\ref{tab:100u_5m_apdex} muestra este resultado.

\begin{table}[!htbp]
    \centering
    \makegapedcells
    \begin{tabular}{|c|c|c|}
    \hline
    APDEX Total & Tolerancia & Frustración\\ \hline
    0.995 & 2 seg & 3 seg \\ \hline
    \end{tabular}
    \caption{APDEX de la prueba con 100 usuarios durante 5 minutos}
    \label{tab:100u_5m_apdex}
\end{table}

\break

\subsection{Pruebas con 300 usuarios simultáneos}
Luego de la primer prueba, se realizó una segunda en la cual 300 usuarios simultáneos se encuentran haciendo pedidos al núcleo, el cual está desplegado en una sola instancia de docker.
Esta prueba duró 10 minutos y los resultados se muestran a continuación.

\subsubsection{Resultados generales de la prueba}

Al igual que en la prueba anterior, jMeter nos muestra estos resultados (~\ref{tab:300u_10m_gen}).

Con un total de 20918 pedidos, el núcleo no pudo resolver en promedio un 1.37\% de los requests dando como resultado el error HTTP 502.

El tiempo promedio de respuesta fue de 4409.80ms., lo cual significa un incremento considerable.

\begin{table}[!htbp]
    \centering
    \makegapedcells
    \begin{tabular}{|c|c|c|c|}
    \hline
    Nombre del Pedido & Cantidad & Errores (\%) & Tiempo Promedio (ms) \\ \hline
    Total & 20918 & 1.37\% & 4409.80\\ \hline
    Carrera | Cantidad graduados & 4175 & 1.20\% & 4242.01\\ \hline
    Carrera | Cantidad ingresantes & 4309 & 1.25\% & 4461.21\\ \hline
    Carrera | Cantidad postulantes & 4242 & 1.65\% & 4641.71\\ \hline
    Carreras & 4070 & 1.25\% & 4274.56\\ \hline
    Plan | Cantidad materias necesarias & 4122 & 1.50\% & 4420.89\\ \hline
    \end{tabular}
    \caption{Pruebas realizadas con 300 usuarios simultáneos durante 10 minutos}
    \label{tab:300u_10m_gen}
\end{table}



\subsubsection{Resultados sobre el uso de recursos durante la prueba}

Como muestra la tabla ~\ref{tab:300u_10m_rec}, hubo un incremento del uso promedio de CPU, aunque el uso de memoria se mantuvo igual.

\begin{table}[!htbp]
    \centering
    \makegapedcells
    \begin{tabular}{|c|c|c|}
    \hline
    Contenedor & CPU & Memoria (MB)\\ \hline
    Núcleo & 88\% & 55 \\ \hline
    \end{tabular}
    \caption{Recursos en promedio consumidos por el contenedor durante la prueba}
    \label{tab:300u_10m_rec}
\end{table}

\subsubsection{APDEX de la prueba}

Luego de ver el tiempo promedio de respuesta de los requests, es esperable que el APDEX baje considerablemente.

En este caso fue de 0.361 (tabla~\ref{tab:300u_10m_apdex}), lo que significa que esa magnitud de usuarios haciendo pedidos no genera conformidad dentro de los parámetros establecidos para la satisfacción de los usuarios.

\begin{table}[!htbp]
    \centering
    \makegapedcells
    \begin{tabular}{|c|c|c|}
    \hline
    APDEX Total & Tolerancia & Frustración\\ \hline
    0.361 & 2 seg & 3 seg \\ \hline
    \end{tabular}
    \caption{APDEX de la prueba con 300 usuarios durante 10 minutos y una instancia de la aplicación}
    \label{tab:300u_10m_apdex}
\end{table}


\subsection{Pruebas con 300 usuarios simultáneos con dos instancias}
Luego de la prueba anterior, donde se refleja una baja considerable del tiempo de respuesta al recibir esa magnitud de usuarios simultáneos, se realizó la misma prueba haciendo uso del balanceador de carga que provee nginx y agregando otra instancia de docker.
Esta prueba duró 10 minutos y los resultados se muestran a continuación.

\subsubsection{Resultados generales de la prueba}

Al igual que en las pruebas anteriores, se tuvo en cuenta los resultados provistos por jMeter (tabla~\ref{tab:300u_10m_2i_gen}).

Estos resultados muestran un incremento en la cantidad de pedidos que pudo resolver el núcleo, pasando de los 20.918 pedidos a 33.499.

Estos resultados también muestran una mejora en el tiempo promedio de respuesta, bajando de 4409.80ms a 2796.50ms. Además, muestran una baja en el promedio de errores pasando de 1.37\% a 0.02\%.

\begin{table}[!htbp]
    \centering
    \makegapedcells
    \begin{tabular}{|c|c|c|c|}
    \hline
    Nombre del Pedido & Cantidad & Errores (\%) & Tiempo Promedio (ms) \\ \hline
    Total & 33499 & 0.02\% & 2796.50\\ \hline
    Carrera | Cantidad graduados & 6691 & 0.01\% & 2844.47\\ \hline
    Carrera | Cantidad ingresantes & 6828 & 0.04\% & 2709.15\\ \hline
    Carrera | Cantidad postulantes & 6756 & 0.04\% & 2945.30\\ \hline
    Carreras & 6584 & 0.02\% & 2782.27\\ \hline
    Plan | Cantidad materias necesarias & 6640 & 0.00\% & 2700.69\\ \hline
    \end{tabular}
    \caption{Pruebas realizadas con 300 usuarios simultáneos durante 10 minutos}
    \label{tab:300u_10m_2i_gen}
\end{table}



\subsubsection{Resultados sobre el uso de recursos durante la prueba}

La tabla~\ref{tab:300u_10m_2i_rec} muestra que el consumo de cpu por instancia bajó de 88\% a 67\% y 68\% para cada una de las respectivas instancias. 

El consumo de memoria se mantuvo en los mismos parámetros, aunque al ser dos instancias esto se duplicó. Igualmente, son valores bajos con respecto a la arquitectura provista.
\begin{table}[!htbp]
    \centering
    \makegapedcells
    \begin{tabular}{|c|c|c|}
    \hline
    Contenedor & CPU & Memoria (MB)\\ \hline
    Núcleo-1 & 67\% & 54 \\ \hline
    Núcleo-2 & 68\% & 54 \\ \hline
    \end{tabular}
    \caption{Recursos en promedio consumidos por los contenedores durante la prueba}
    \label{tab:300u_10m_2i_rec}
\end{table}
\subsubsection{APDEX de la prueba}
\begin{table}[!htbp]
    \centering
    \makegapedcells
    \begin{tabular}{|c|c|c|}
    \hline
    APDEX Total & Tolerancia & Frustración\\ \hline
    0.667 & 2 seg & 3 seg \\ \hline
    \end{tabular}
    \caption{APDEX de la prueba con 300 usuarios durante 10 minutos y dos instancias de la aplicación}
    \label{tab:300u_10m_2i_apdex}
\end{table}
\break
\section{Conclusiones sobre las pruebas}

Luego de realizar diferentes pruebas al núcleo, se observa que con 100 usuarios simultáneos haciendo requests constantes durante 5 minutos, llegando a un total de 16977 pedidos, la aplicación respondió a la perfección llegando a un APDEX de 0.995 y 0\% de errores.

Después, se planteó un escenario donde 300 usuarios simultáneos realizaron requests constantes durante 10 minutos (20918 pedidos) y se pudo observar una baja considerable del APDEX llegando a 0.361. Los errores también aumentaron llegando a 1.37\%. 

Por último, y con motivo de mejorar los resultados para una mejor experiencia de los usuarios, se realizó la misma prueba duplicando las instancias de docker y haciendo uso del balanceador de carga de Nginx. Los resultados mejoraron notablemente llegando a un APDEX de 0.667 y bajando los errores a 0.02\%. Además, se incrementó la cantidad de pedidos que resolvió en ese tiempo, llevandolos a 33499 (un 60.1\% más).

Teniendo en cuenta que en la UNQ los momentos en que los sistemas académicos son mas exigidos es en el inicio y en el final de los cuatrimestres, se podría duplicar las instancias de docker en estos momentos (o triplicar, de ser necesario).

		\chapter{Conclusiones}
\label{sec:conclusiones}

\section[Conclusiones]{Conclusiones}

A lo largo de esta tesis se han investigado numerosas herramientas de software para construir una solución para la centralización, disposición, análisis y visualización de datos académicos.

En primer lugar, se pensó una solución como servicios independientes, intercambiables y escalables según la necesidad. Esto facilita que si en un futuro aparece una tecnología superadora para una tarea en particular, sólo se tenga que reemplazar ese servicio.

Se ha logrado crear un Núcleo que almacena datos y los entrega a través de una API REST, la cual puede ser consmumida por usuarios que tengan los permisos correspondientes.

Se creó un servicio para el análisis de los datos consultados al núcleo, el cual también entrega el resultado del procesamiento de esos datos para que puedan ser consumidos.

Luego, se creó un servicio que facilita la visualización de los datos procesados. Éste consume los datos del módulo de análisis y muestra los resultados en forma de gráfico de columnas, gráfico de tortas, gráficos de radar, gráficos de puntos, tablas, etc.

Este conjunto de servicios les permite a las partes interesadas consultar la información, modificarla, consultar métricas sobre carreras, materias y estudiantes de forma individual o colectiva y consultar datos sobre períodos en particular.
Esta solución se ajusta perfectamente a la arquitectura provista por la Universidad.

Se ha aprendido acerca de la importancia de la disposición de los datos y cómo esto puede ayudar a tomar decisiones tempranas sobre la vida académica de la Universidad Nacional de Quilmes, sus carreras, sus materias y su alumnado.


\section[Lineas de Trabajos futuros]{Lineas de Trabajos futuros}

Dado que el sistema se pensó de forma tal que se pueda extender fácilmente, y que sus datos puedan ser consultados por otros servicios, existen varias mejoras y puntos de extensión que se detallan a continuación:

\subsection[Aplicación para estudiantes]{Aplicación para estudiantes}

Podría existir una aplicación para que usen los estudiantes (puede ser móvil o web), donde puedan consultar su historial dentro de la Universidad. Si bien esta tarea se puede realizar en Guaraní, se podría integrar con servicios que brinda la universidad, como la consulta del saldo disponible en la tarjeta de la UNQ, notificaciones con novedades y noticias que estén relacionadas únicamente con los estudiantes.
Esta aplicación, a su vez, puede tener un aspecto mas social: que los estudiantes pongan tips y consejos sobre las materias.
Además, podría recibir feedback sobre los que ya cursaron, los horarios, las comodidades, la calidad de enseñanza, etc.

\subsection[Integración con Guaraní]{Integración con Guaraní}

Sería muy conveniente poder conectar el Núcleo con Guaraní a través de una API web. En la actualidad no existe dicha API, pero con el lanzamiento de Guaraní 3 debería ser posible. De esta forma, no se deberían importar más los datos a través de planillas, sino que se haría de forma automática a través del Núcleo.

\subsection[Integración con sistema de encuestas de inscripción]{Integración con sistema de encuestas de inscripción}

El sistema actual que utilizan algunas carreras de la UNQ para consultar las posibles inscripciones de los estudiantes debería ser integrado al núcleo para que consuma datos, y luego debería enviar los resultado para luego con esa información mostrar métricas a los directores de carreras. 


\subsection[GraphQL en el núcleo]{GraphQL en el núcleo}

GraphQL es un lenguaje de queries para APIs, donde los clientes definen precisamente qué datos quieren en lugar de recibir todos los datos disponibles. Esto significaría una reducción en el tamaño de respuesta de la API del núcleo.

\subsection[Proveedor de identidad (IDP)]{Proveedor de identidad (IDP)}

Dado que el potencial de extensión es amplio, una buena forma de desligarle responsabilidades al núcleo es crear un proveedor de identidad. En lugar de los usuarios identificarse ante el núcleo y que este les provea un token, debería hacerlo un servicio independiente que tenga conocimiento de todos los usuarios, y éste se encargue de validar quién puede ingresar y quién no, y qué permisos tiene.

\subsection[Acceso Centralizado]{Acceso Centralizado}

Una vez que exista el IDP, sería muy útil un acceso centralizado. Esto significa que, una vez que el usuario se identifica con el IDP, éste lo considera *logueado* en todas las aplicaciones que el usuario tiene acceso. Cuando ingrese a otra aplicación, automáticamente el IDP lo deja ingresar.




	\backmatter
		\bibliographystyle{IEEEtran}
		%\bibliographystyle{plain}     %You may prefer \bibliographystyle{alpha}
		%\bibliographystyle{alpha}
		%\bibliographystyle{babalpha}
		\bibliography{books}
		\nocite{*}
		\chapter{Anexos}

\section{Pantallas básicas del Núcleo}


\subsubsection{Login}
\begin{figure}[!htbp]
  \centering
    \includegraphics[scale=0.3]{images/nucleo/nucleo-login.png}
  \captionof{figure}{Pantalla de login}
  \label{fig:nucleo-login}
\end{figure}

\subsubsection{Home}
\begin{figure}[!htbp]
  \centering
    \includegraphics[scale=0.3]{images/nucleo/nucleo-home.png}
  \captionof{figure}{Pantalla principal del admin}
  \label{fig:nucleo-home}
\end{figure}

\subsubsection{Importadores}
\begin{figure}[!htbp]
  \centering
    \includegraphics[scale=0.3]{images/nucleo/nucleo-importador.png}
  \captionof{figure}{Pantalla de importadores}
  \label{fig:nucleo-importador}
\end{figure}

\subsubsection{Listado}
\begin{figure}[!htbp]
  \centering
    \includegraphics[scale=0.3]{images/nucleo/nucleo-list.png}
  \captionof{figure}{Pantalla de listados}
  \label{fig:nucleo-listado}
\end{figure}

\subsubsection{Edicion}
\begin{figure}[!htbp]
  \centering
    \includegraphics[scale=0.3]{images/nucleo/nucleo-edit.png}
  \captionof{figure}{Pantalla de edición}
  \label{fig:nucleo-edicion}
\end{figure}

\subsubsection{Pedir Token}
\begin{table}[!htbp]
    \centering
    \makegapedcells
    \begin{tabular}{|c|c|c|c|c|}
    \hline
    URI & Método & Parámetros & Content-Type \\ \hline
    api/token/ & POST & username,password & application/x-www-form-urlencoded \\ \hline
    \end{tabular}
    \caption{Método para pedir un token al núcleo}
    \label{tab:tabla_token}
\end{table}

\subsubsection{API}
\begin{table}[!htbp]
    \centering
    \makegapedcells
    \begin{tabular}{|c|c|c|}
    \hline
    URI & Método & Autorización\\ \hline
    carreras/<str:cc>/alumnos/ & GET & Bearer  \\ \hline
    carreras/<str:cc>/alumnos-completos/ & GET & Bearer  \\ \hline
    carreras/<str:cc>/planes/<int:plan\_anio>/ & GET & Bearer  \\ \hline
    carreras/<str:cc>/planes/<int:pa>/cantidad-materias-necesarias/ & GET & Bearer  \\ \hline
    carreras/<str:cc>/planes/ & GET & Bearer  \\ \hline
    carreras/<str:cc>/materiascursadas/ & GET & Bearer  \\ \hline
    carreras/<str:cc>/inscripciones/ & GET & Bearer  \\ \hline
    carreras/<str:cc>/cantidad-graduados/ & GET & Bearer  \\ \hline
    carreras/<str:cc>/cantidad-graduados/<int:anio>/ & GET & Bearer  \\ \hline
    carreras/<str:cc>/cantidad-cursantes/ & GET & Bearer  \\ \hline
    carreras/<str:cc>/cantidad-cursantes/<int:anio>/ & GET & Bearer  \\ \hline
    carreras/<str:cc>/cantidad-ingresantes/ & GET & Bearer  \\ \hline
    carreras/<str:cc>/cantidad-ingresantes/<int:anio>/ & GET & Bearer  \\ \hline
    carreras/<str:cc>/cantidad-postulantes/<int:anio>/ & GET & Bearer  \\ \hline
    carreras/<str:cc>/cantidad-postulantes/ & GET & Bearer  \\ \hline
    alumno/<str:legajo>/cursadas/ & GET & Bearer  \\ \hline
    alumno/<str:legajo>/inscripciones/ & GET & Bearer  \\ \hline
    materia/<str:codigo>/alumnos/ & GET & Bearer  \\ \hline
    \end{tabular}
    \caption{Métodos para pedir datos. (cc: código de carrera, pa: plan año)}
    \label{tab:tabla_api}
\end{table}

\section{Pantallas básicas de Seguimiento Académico}

\subsubsection{Login}
\begin{figure}[!htbp]
  \centering
    \includegraphics[scale=0.3]{images/seguimiento-academico/sa-login.png}
  \captionof{figure}{Login de Seguimiento Académico}
  \label{fig:sa-login}
\end{figure}

\subsubsection{Home}
\begin{figure}[!htbp]
  \centering
    \includegraphics[scale=0.3]{images/seguimiento-academico/sa-home.png}
  \captionof{figure}{Pantalla principal de Seguimiento Académico}
  \label{fig:sa-home}
\end{figure}

\subsubsection{Reporte de Carrera}
\begin{figure}[!htbp]
  \centering
    \includegraphics[scale=0.3]{images/seguimiento-academico/sa-form-carrera.png}
  \captionof{figure}{Pantalla Reporte de Carrera de Seguimiento Académico}
  \label{fig:sa-carrera}
\end{figure}

\subsubsection{Reporte de Materia}
\begin{figure}[!htbp]
  \centering
    \includegraphics[scale=0.3]{images/seguimiento-academico/sa-form-materia.png}
  \captionof{figure}{Pantalla Reporte de Materia de Seguimiento Académico}
  \label{fig:sa-materia}
\end{figure}

\subsubsection{Reporte de Alumno}
\begin{figure}[!htbp]
  \centering
    \includegraphics[scale=0.3]{images/seguimiento-academico/sa-form-alumno.png}
  \captionof{figure}{Pantalla Reporte de Alumno de Seguimiento Académico}
  \label{fig:sa-alumno}
\end{figure}	%opcional
\end{document}