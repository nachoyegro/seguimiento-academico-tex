\chapter{Introducción}
\label{sec:introduccion}


El déficit en ingenieros y graduados en las llamadas disciplinas STEM (Ciencia, Tecnología, Ingeniería y Matemática) es un problema que no sólo afecta a nuestro país, sino que se manifiesta como una problemática global. Particularmente en Argentina, cada año quedan sin cubrir 5.000 puestos en la industria del software por falta de profesionales (según la Cámara de la Industria Argentina del Software). El sector emplea a 90.000 personas y representa una de las principales exportaciones de valor agregado con un crecimiento del 10\% anual, pero “la matrícula en carreras de sistemas quedó estancada en 20.000 y se reciben 4.000 por año, cuando la industria requiere el doble“ (según Fundación Sadosky).
Bajo este contexto, se implementaron la Tecnicatura en Programación Informática en 2003 y la Licenciatura en Informática en 2012 en la Universidad Nacional de Quilmes para desarrollar profesionales altamente calificados y ofrecer carreras distintivas de aquellas ofertadas por los centros universitarios próximos (UBA, UNLP, etc) y que apunten a cubrir necesidades sociales e industriales concretas.
 
Desde la creación de ambas carreras, se ha dado un crecimiento constante  en sus respectivas matrículas. A partir del 2015, ese crecimiento es más notable en la UNQ pues se incorporaron a sus planes de estudio las materias del Ciclo Ingreso como parte de las materias de la carrera, denominándolo como Ciclo Introductorio. A partir de este cambio, los estudiantes ya eran parte de la carrera cuando se incorporaban a la UNQ, lo que no ocurría en años previos. Los números del crecimiento se pueden ver en las siguientes tablas:


\begin{table}[!htbp]
    \centering
    \begin{tabular}{|c|c|c|c|}
    \hline
    Año & Estudiantes & Total Matrícula & Tasa Crecimiento \\
    \hline
    2013 & 23 & 23 & \\
    \hline
    2014 & 94 & 117 & 5,09 \\ 
    \hline
    2015 & 97 & 214 & 1,83 \\
    \hline
    2016 & 333 & 547 & 2,56 \\
    \hline
    2017 & 323 & 870 & 1,59 \\ 
    \hline
    2018 & 297 & 1167 & 1,34 \\ 
    \hline
    \end{tabular}
    \caption{Crecimiento de la Matrícula en la Licenciatura en Informática}
    \label{tab:tabla_planes}
\end{table}



\begin{table}[!htbp]
    \centering
    \begin{tabular}{|c|c|c|c|}
    \hline
    Año & Estudiantes & Total Matrícula & Tasa Crecimiento \\
    \hline
    2003 & 1 & 1 & \\
    \hline
    2004 & 3 & 4 & 4,00 \\
    \hline
    2005 & 4 & 8 & 2,00 \\
    \hline
    2006 & 0 & 8 & 1,00 \\
    \hline
    2007 & 41 & 49 & 6,13 \\
    \hline
    2008 & 106 & 155 & 3,16 \\
    \hline
    2009 & 99 & 254 & 1,64 \\
    \hline
    2010 & 108 & 362 & 1,43 \\
    \hline
    2011 & 142 & 504 & 1,39 \\ 
    \hline
    2012 & 112 & 616 & 1,22 \\
    \hline
    2013 & 166 & 782 & 1,27 \\
    \hline
    2014 & 103 & 885 & 1,13 \\
    \hline
    2015 & 88 & 973 & 1,10 \\
    \hline
    2016 & 287 & 1260 & 1,29 \\
    \hline
    2017 & 295 & 1555 & 1,23 \\
    \hline
    2018 & 271 & 1826 & 1,17 \\
    \hline
    \end{tabular}
    \caption{Crecimiento de la Matrícula en la Tecnicatura en Programación Informática}
    \label{tab:tabla_planes}
\end{table}

Desde sus inicios, la dirección de ambas carreras ha creado y mantenido un registro de datos personales de los estudiantes, las materias cursadas y las materias que tienen intenciones de cursar cuatrimestre a cuatrimestre. Con este registro, inicialmente se ha permitido organizar las intenciones de inscripción de cada estudiante, generar las listas de emails con las cuales trabaja cada materia, manejar el cupo de las comisiones de las materias para evitar superpoblación en las mismas, detectar posibles candidatos a presentarse como auxiliares académicos, detectar diferentes problemáticas en la cursada de los estudiantes (sobretodo en las instancias finales de la carrera) y realizar diferentes análisis de la evolución de la carrera de los estudiantes durante cada cuatrimestre, entre otras.
En la actualidad, todos los análisis descritos previamente se realizan en base a planillas Excel que mantienen toda la información de los estudiantes y materias. Esta información ha sido obtenida ya sea a través de los estudiantes mismos o bien, usando el sistemas SIU-Guaraní de la UNQ.
Con el incremento de la matrícula año a año, este proceso de análisis “ad-hoc” se hace más complejo y lleva demasiado tiempo. Las direcciones de las carreras de informática trabajan alrededor de una semana a tiempo completo post inscripciones y terminadas las cursadas para hacer un procesamiento adecuado con  la información de los estudiantes (sobretodo aquellos que comenzaron el Ciclo Básico de cualquiera de las carreras). Durante el cuatrimestre, el sistema de registros se consulta constantemente para generar diferentes estadísticas para planificar la evolución de la carrera. Aún cuando toda la información está disponible y analizada, todo este proceso es ineficiente y demora decisiones relevantes a la carrera, en lo que respecta tanto a estudiantes como a docentes. Los problemas actuales (entre otros) son:
Como la información se guarda en planilla “Excel”, no existe la posibilidad de tener varios usuarios (directores, asistentes de carrera, docentes) colaborando en forma cooperativa (distribuida).
Los datos provistos por SIU-Guaraní y los resultados de cada materia al finalizar cada cuatrimestre se incorporan de forma manual en los registros.
Los análisis se realizan con fórmulas analizando las planillas y su información (ejemplo, generación de estadísticas)
No existe una forma automática de controlar la información que tiene SIU-Guaraní acerca de los estudiantes y su situación de sus cursadas en las carreras.

\section[Objetivo general]{Objetivo general}
En base a esta lista de problemas que se enumeran en la Introducción, este plan de trabajo plantea como objetivo general el desarrollo de una aplicación de seguimiento académico que puede ayudar en el proceso de gestión y decisión de la Dirección de Carreras de TPI/LI, con las siguientes acciones:
Analizar información de los estudiantes y de las materias que cursa o debe cursar de la TPI y de la LI, desde su incorporación a la carrera hasta el final de la carrera    
Generar recomendaciones para los estudiantes acerca de sus elecciones de materia a principios de cada cuatrimestre
Generar estadísticas y visualizaciones relevantes que ayuden a la Dirección de Carreras a tomar decisiones durante el año académico.

\section[Objetivos específicos]{Objetivos específicos}
Para el desarrollo de la aplicación de seguimiento académico, los objetivos específicos en base a la implementación de la plataforma serán los siguientes:

\begin{outline}
    \1 Relevamiento de los requerimientos y casos de uso.
    \1 Relevamiento del hardware disponible en la Universidad.
    \1 Diseño del software y la arquitectura del mismo.
    \1 Desarrollo de software.
    \1 Tests de unidad e integración.
    \1 Pruebas de stress.
    \1 Documentación.
    \1 Puesta en producción.
\end{outline}