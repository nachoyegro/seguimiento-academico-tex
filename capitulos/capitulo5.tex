\chapter{Pruebas realizadas}
\label{sec:implementacion}

\section[Pruebas y resultados]{Pruebas y resultados}

\subsection[Resultados esperados]{Resultados esperados}

El desarrollo de las aplicaciones debe darle respuesta a todos los interesados. Para que se cumplan estos requisitos, la implementación tiene que ser capaz de manejar todos los pedidos que recibe y responder en consecuencia. 

Si bien la problemática inicial implica que sólo se consultarán datos para analizarlos y ser visualizados por los directores de las carreras de la Universidad Nacional de Quilmes, no hay que descartar que pueda ser extendido dado su potencial.

Para garantizar que todo funcione de manera correcta se realizaron diferentes pruebas con escenarios cambiantes.
Para todas las pruebas se eligieron 5 URIs y se realizaron sobre una base de datos con 1.424 estudiantes y 13.234 materias cursadas por estos estudiantes.
La arquitectura utilizada para las pruebas se asemeja a la que dispondrá a la hora de la implementación.

Se utilizó la herramienta jMeter para realizar los requests y evaluar los tiempos de respuesta y errores, entre otras métricas.

Para medir los tiempos de respuesta, se usará APDEX.

APDEX (Índice de Performance de la Aplicación, por sus siglas en inglés) es un estándar abierto para medir la \textit{performance} de una a aplicación de software. Su propósito es convertir las mediciones en información sobre la satisfacción del usuario, al especificar una forma uniforme de analizar e informar sobre el grado en que el rendimiento medido cumple con las expectativas del usuario.
Para realizar una medición \textit{APDEX}, es necesario establecer un tiempo \emph{satisfactorio} y un tiempo \emph{tolerable}.
Una vez establecidos estos valores y teniendo los resultados de la prueba, se calcula el índice:

\begin{align*}
  Apdex = \frac{Satisfactorios + \frac{Tolerables}{2}}{Total}\\
\end{align*}

\subsection{Pruebas con 100 usuarios simultáneos}
En primer lugar, se realizó una prueba en la cual 100 usuarios simultáneos se encuentran haciendo pedidos al núcleo, el cual está desplegado en una sola instancia de docker.
Esta prueba duró 5 minutos y los resultados se muestran a continuación.


\subsubsection{Resultados generales de la prueba}


Los resultados generales, como muestra la Tabla ~\ref{tab:100u_5m_gen}, indican que durante los 5 minutos que duró la prueba se realizaron 16.997 requests en total, y el núcleo pudo responder sin ningún error.

El tiempo promedio de respuesta fue de 1043.79ms.

\begin{table}[!htbp]
    \centering
    \makegapedcells
    \begin{tabular}{|c|c|c|c|}
    \hline
    Nombre del Pedido & Cantidad & Errores (\%) & Tiempo Promedio (ms) \\ \hline
    Total & 16977 & 0.00\% & 1043.79\\ \hline
    Carrera | graduados & 2834 & 0.00\% & 1038.98\\ \hline
    Carrera | ingresantes & 2871 & 0.00\% & 1057.51\\ \hline
    Carrera | postulantes & 2850 & 0.00\% & 1046.61\\ \hline
    Carreras & 2793 & 0.00\% & 1851.00\\ \hline
    Plan | Materias necesarias & 5629 & 0.00\% & 1036.10\\ \hline
    
    \end{tabular}
    \caption{Pruebas realizadas con 100 usuarios simultáneos durante 5 minutos}
    \label{tab:100u_5m_gen}
\end{table}


\subsubsection{Resultados sobre el uso de recursos durante la prueba}

Para evaluar el consumo de recursos, se utilizó el comando \textit{docker stats} que realiza una transmisión en vivo de los recursos que van consumiendo los contenedores.
En promedio utilizó un 68\% del CPU y 55MB de memoria RAM (Tabla ~\ref{tab:100u_5m_rec}).


\begin{table}[!htbp]
    \centering
    \makegapedcells
    \begin{tabular}{|c|c|c|}
    \hline
    Contenedor & CPU & Memoria (MB)\\ \hline
    Núcleo & 68\% & 55 \\ \hline
    \end{tabular}
    \caption{Recursos en promedio consumidos por el contenedor durante la prueba}
    \label{tab:100u_5m_rec}
\end{table}


\subsubsection{APDEX de la prueba}

Para determinar el APDEX de la prueba, se estableció que la \textit{tolerancia} es de 2 segundos, y la \textit{frustración} es de 3 segundos. 

Como se explicó anteriormente, esta prueba tiene en cuenta el tiempo que tardan todos los requests para luego determinar este valor.

Para esta prueba, el APDEX fue de 0.995, lo cual es casi perfecto (el valor máximo es 1) dentro de los parámetros elegidos.

La Tabla ~\ref{tab:100u_5m_apdex} muestra este resultado.

\begin{table}[!htbp]
    \centering
    \makegapedcells
    \begin{tabular}{|c|c|c|}
    \hline
    APDEX Total & Tolerancia & Frustración\\ \hline
    0.995 & 2 seg & 3 seg \\ \hline
    \end{tabular}
    \caption{APDEX de la prueba con 100 usuarios durante 5 minutos}
    \label{tab:100u_5m_apdex}
\end{table}

\break

\subsection{Pruebas con 300 usuarios simultáneos}
Luego de la primer prueba, se realizó una segunda en la cual 300 usuarios simultáneos se encuentran haciendo pedidos al núcleo, el cual está desplegado en una sola instancia de docker.
Esta prueba duró 10 minutos y los resultados se muestran a continuación.

\subsubsection{Resultados generales de la prueba}

Al igual que en la prueba anterior, jMeter nos muestra estos resultados (Tabla ~\ref{tab:300u_10m_gen}).

Con un total de 20918 pedidos, el núcleo no pudo resolver en promedio un 1.37\% de los requests dando como resultado el error HTTP 502.

El tiempo promedio de respuesta fue de 4409.80ms., lo cual significa un incremento considerable.

\begin{table}[!htbp]
    \centering
    \makegapedcells
    \begin{tabular}{|c|c|c|c|}
    \hline
    Nombre del Pedido & Cantidad & Errores (\%) & Tiempo Promedio (ms) \\ \hline
    Total & 20918 & 1.37\% & 4409.80\\ \hline
    Carrera | Cantidad graduados & 4175 & 1.20\% & 4242.01\\ \hline
    Carrera | Cantidad ingresantes & 4309 & 1.25\% & 4461.21\\ \hline
    Carrera | Cantidad postulantes & 4242 & 1.65\% & 4641.71\\ \hline
    Carreras & 4070 & 1.25\% & 4274.56\\ \hline
    Plan | Cantidad materias necesarias & 4122 & 1.50\% & 4420.89\\ \hline
    \end{tabular}
    \caption{Pruebas realizadas con 300 usuarios simultáneos durante 10 minutos}
    \label{tab:300u_10m_gen}
\end{table}



\subsubsection{Resultados sobre el uso de recursos durante la prueba}

Como muestra la Tabla ~\ref{tab:300u_10m_rec}, hubo un incremento del uso promedio de CPU, aunque el uso de memoria se mantuvo igual.

\begin{table}[!htbp]
    \centering
    \makegapedcells
    \begin{tabular}{|c|c|c|}
    \hline
    Contenedor & CPU & Memoria (MB)\\ \hline
    Núcleo & 88\% & 55 \\ \hline
    \end{tabular}
    \caption{Recursos en promedio consumidos por el contenedor durante la prueba}
    \label{tab:300u_10m_rec}
\end{table}

\subsubsection{APDEX de la prueba}

Luego de ver el tiempo promedio de respuesta de los requests, es esperable que el APDEX baje considerablemente.

En este caso fue de 0.361 (Tabla~\ref{tab:300u_10m_apdex}), lo que significa que esa magnitud de usuarios haciendo pedidos no genera conformidad dentro de los parámetros establecidos para la satisfacción de los usuarios.

\begin{table}[!htbp]
    \centering
    \makegapedcells
    \begin{tabular}{|c|c|c|}
    \hline
    APDEX Total & Tolerancia & Frustración\\ \hline
    0.361 & 2 seg & 3 seg \\ \hline
    \end{tabular}
    \caption{APDEX de la prueba con 300 usuarios durante 10 minutos y una instancia de la aplicación}
    \label{tab:300u_10m_apdex}
\end{table}


\subsection{Pruebas con 300 usuarios simultáneos con dos instancias}
Luego de la prueba anterior, donde se refleja una baja considerable del tiempo de respuesta al recibir esa magnitud de usuarios simultáneos, se realizó la misma prueba haciendo uso del balanceador de carga que provee nginx y agregando otra instancia de docker.
Esta prueba duró 10 minutos y los resultados se muestran a continuación.

\subsubsection{Resultados generales de la prueba}

Al igual que en las pruebas anteriores, se tuvo en cuenta los resultados provistos por jMeter (Tabla~\ref{tab:300u_10m_2i_gen}).

Estos resultados muestran un incremento en la cantidad de pedidos que pudo resolver el núcleo, pasando de los 20.918 pedidos a 33.499.

Estos resultados también muestran una mejora en el tiempo promedio de respuesta, bajando de 4409.80ms a 2796.50ms. Además, muestran una baja en el promedio de errores pasando de 1.37\% a 0.02\%.

\begin{table}[!htbp]
    \centering
    \makegapedcells
    \begin{tabular}{|c|c|c|c|}
    \hline
    Nombre del Pedido & Cantidad & Errores (\%) & Tiempo Promedio (ms) \\ \hline
    Total & 33499 & 0.02\% & 2796.50\\ \hline
    Carrera | Cantidad graduados & 6691 & 0.01\% & 2844.47\\ \hline
    Carrera | Cantidad ingresantes & 6828 & 0.04\% & 2709.15\\ \hline
    Carrera | Cantidad postulantes & 6756 & 0.04\% & 2945.30\\ \hline
    Carreras & 6584 & 0.02\% & 2782.27\\ \hline
    Plan | Cantidad materias necesarias & 6640 & 0.00\% & 2700.69\\ \hline
    \end{tabular}
    \caption{Pruebas realizadas con 300 usuarios simultáneos durante 10 minutos}
    \label{tab:300u_10m_2i_gen}
\end{table}



\subsubsection{Resultados sobre el uso de recursos durante la prueba}

La Tabla~\ref{tab:300u_10m_2i_rec} muestra que el consumo de CPU por instancia bajó de 88\% a 67\% y 68\% para cada una de las respectivas instancias. 

El consumo de memoria se mantuvo en los mismos parámetros, aunque al ser dos instancias el consumo se duplicó. Igualmente, son valores bajos con respecto a la arquitectura provista.
\begin{table}[!htbp]
    \centering
    \makegapedcells
    \begin{tabular}{|c|c|c|}
    \hline
    Contenedor & CPU & Memoria (MB)\\ \hline
    Núcleo-1 & 67\% & 54 \\ \hline
    Núcleo-2 & 68\% & 54 \\ \hline
    \end{tabular}
    \caption{Recursos en promedio consumidos por los contenedores durante la prueba}
    \label{tab:300u_10m_2i_rec}
\end{table}
\subsubsection{APDEX de la prueba}
\begin{table}[!htbp]
    \centering
    \makegapedcells
    \begin{tabular}{|c|c|c|}
    \hline
    APDEX Total & Tolerancia & Frustración\\ \hline
    0.667 & 2 seg & 3 seg \\ \hline
    \end{tabular}
    \caption{APDEX de la prueba con 300 usuarios durante 10 minutos y dos instancias de la aplicación}
    \label{tab:300u_10m_2i_apdex}
\end{table}
\break
\section{Conclusiones sobre las pruebas}

Luego de realizar diferentes pruebas al núcleo, se observa que con 100 usuarios simultáneos haciendo requests constantes durante 5 minutos, llegando a un total de 16977 pedidos, la aplicación respondió a la perfección llegando a un APDEX de 0.995 y 0\% de errores.

Después, se planteó un escenario donde 300 usuarios simultáneos realizaron requests constantes durante 10 minutos (20918 pedidos) y se pudo observar una baja considerable del APDEX llegando a 0.361. Los errores también aumentaron llegando a 1.37\%. 

Por último, y con motivo de mejorar los resultados para una mejor experiencia de los usuarios, se realizó la misma prueba duplicando las instancias de docker y haciendo uso del balanceador de carga de Nginx. Los resultados mejoraron notablemente llegando a un APDEX de 0.667 y bajando los errores a 0.02\%. Además, se incrementó la cantidad de pedidos que resolvió en ese tiempo, llevandolos a 33499 (un 60.1\% más).

Teniendo en cuenta que en la UNQ los momentos en que los sistemas académicos son más exigidos es en el inicio y en el final de los cuatrimestres, se podría duplicar las instancias de docker en estos momentos (o triplicar, de ser necesario).
