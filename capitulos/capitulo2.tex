\chapter{Estado del arte}
\label{sec:hello}

\section[Crecimiento exponencial de los datos]{Crecimiento exponencial de los datos} 


La humanidad, hasta el año 2005, creó 130 ExaBytes de datos (un exabyte son 1.073.741.824 GB). Para el año 2010, la cantidad de datos subió a 1200 exabytes. Para el año 2015, la cantidad de datos subió a 7900 exabytes. Y para el año 2020, se estima que esa cantidad llegó a los 40900 exabytes \cite{IDC}.

En resumen, la cantidad de datos que producimos los humanos está creciendo exponencialmente, y las universidades no son la excepción.



\section[El rol de las universidades con respecto a los datos]{El rol de las universidades con respecto a los datos}

Hoy en día, se está transformando más en una norma que en una excepción para las universidades del mundo utilizar los datos que tienen sobre los estudiantes para propósitos académicos. Existen universidades que usan estos datos para ayudar a los estudiantes a tener éxito, o para ofrecerles ayuda cuando se considera que no están rindiendo sus exámenes, o bien para mantenerlos cursando y seguir recibiendo ingresos, en el caso de las instituciones privadas.
El análisis de los datos se hace cada vez más frecuente y necesario, y de puede ser más eficiente ya que las herramientas de análisis fueron mejorando.

En las secciones siguientes se describen el caso de 3 universidades que han publicado sus estrategias de análisis con respecto a sus estudiantes.

\subsection[Universidad de San Francisco]{Universidad de San Francisco}

En la Universidad de San Francisco (USF), los datos de los estudiantes tales como asistencia, notas, y materias son usados para determinar su progreso. Cuando observan que un estudiante está teniendo dificultades, usando la tecnología desarrollada por el departamento de sistemas de USF, alertan a los tutores para que hablen con el estudiante.
Además, usan una aplicación móvil llamada USFMobile, que consulta datos de sus sistemas y generan alertas.
Eventualmente, esos datos serán usados para \textit{“entender qué hacen los estudiantes, y tratar de ayudarlos”} (Opinder Bawa, Vice presidente y \textit{Chief Information Officer})

\subsection[Universidad de Missouri]{Universidad de Missouri}
La Universidad de Missouri arma cuestionarios de 12 preguntas para los estudiantes. El propósito es conocer cómo los estudiantes se desenvuelven en términos académicos, sociales y financieros. En caso que un estudiante indique alguna inquietud, esa información es utilizada para tener un mayor impacto. Los datos son almacenados en los sistemas de la universidad.
Los tres temas principales que mostró la encuesta, están relacionados con las dificultades de algunos cursos, la asistencia y las dificultades financieras de los estudiantes. Estos datos le dieron a la Universidad un panorama y pudo actuar rápidamente en consecuencia (Ashli Grabau, director de Iniciativas Estratégicas y Evaluación para Asuntos Estudiantiles)

\subsection[Universidad Estatal de Georgia]{Universidad Estatal de Georgia}

La Universidad Estatal de Georgia se convirtió en un ejemplo a nivel nacional en Estados Unidos, por el papel que adaptó para ayudar a los estudiantes de bajos recursos y los de primera generación. 
Esta universidad usa análisis predictivos para identificar a los estudiantes que podrían abandonar o no llegar a recibirse. Esto lo logran usando información pasada, la cual es utilizada para predecir futuros eventos. 
La Universidad de Georgia analizó 2 millones y medio de calificaciones obtenidas por los estudiantes a lo largo de 10 años para crear una lista de factores que predecían qué estudiantes son menos probables a graduarse. Este sistema tiene más de 800 alertas, apuntadas a indicarle a los tutores acerca de sus estudiantes y así poder ayudarlos. Por ejemplo, un tutor recibe un alerta cuando un estudiante no recibe una nota satisfactoria en una materia que es fundamental para su formación. 

\section[Contexto de las Universidades en Argentina]{Contexto de las Universidades en Argentina}

Las Universidades Nacionales (públicas o privadas) guardan sus datos a traves del Sistema de Información Universitaria (www.siu.edu.ar).

Este sistema se compone de un grupo de aplicaciones informáticas que colaboran en la gestión y la calidad de los datos que se producen en el ámbito de cada universidad.

El SIU tiene en la actualidad las siguientes aplicaciones:

\begin{outline}
    \2 Completar
\end{outline}

De las aplicaciones enumeradas, detallamos a continuación aquellas que influyen en nuestra aplicación:

\begin{outline}
    \1 SIU Guaraní que se utiliza para la administración de las tareas académicas para obtener información consistente para niveles operativos y directivos. Las funcionalidades actuales se detallan en www.siu.edu.ar/siu-guarani
    \1 SIU Kolla permite la generación de encuestas para realizar relevamientos sobre distintas problemáticas del ámbito universitario. Las funcionalidades actuales se detallan en www.siu.edu.ar/siu-kolla
    \1 SIU Araucano permite informar estadísticas de ingreso, regularidad y egreso de los estudiantes. Además, procesa las cifras de la oferta educativa, como las cantidades de alumnos por materia, materias aprobadas por alumno, materias ofertadas o la antigüedad de los alumnos. Las funcionalidades actuales se detallan en www.siu.edu.ar/siu-araucano
    \1 SIU Wichi permite visualizar y analizar de manera integrada los datos históricos de ejecución presupuestaria, académicos de personal y patrimonio, buscando como objetivo colaborar con las decisiones que tomen los distintos actores de la organización, sustentadas sobre una base de conocimiento. Las funcionalidades actuales se detallan en www.siu.edu.ar/siu-wichi
\end{outline}

\section[El rol de la Universidad Nacional de Quilmes]{El rol de la Universidad Nacional de Quilmes}

Como se mencionó previamente, la Universidad Nacional de Quilmes (como otras universidades) tiene en la actualidad todo el historial de las carreras, sus estudiantes, sus materias cursadas y las notas obtenidas. La Universidad también cuenta con otros datos referentes a su funcionamiento como institución educativa, pero sólo mencionamos aquellos que son relevantes para el desarrollo de este informe y la aplicación correspondiente.

Todos los datos mencionados previamente están guardados en diferentes módulos del SIU (descriptos en la sección previa).

Debemos remarcar que los módulos de SIU generan datos o estadísticas generales de las carreras, y que nuestro trabajo tiene como propósito usar estos datos para tomar decisiones de las problemáticas propias de las carreras de informática de la UNQ.



\section[Planillas de datos]{Planillas de datos}

La información solicitada a la Secretaría de Gestión Académica se obtiene a través de planillas Excel. A continuación listamos las diferentes planillas y sus formatos.

\subsection[Planes de estudio]{Planes de estudio}

Los datos de planes de estudio contienen información de las materias junto con el área a la que pertenecen, el núcleo, y los créditos, entre otros.
Las columnas de la planilla presentan el formato detallado en la Tabla \ref{tab:tabla_planes}.

\begin{table}[!htbp]
    \centering
    \begin{tabular}{|c|c|c|c|c|c|c|c|}
    \hline
    Carrera & Plan & Cuat. & Núcleo & Área & Materia & Créd. & Nombre \\ \hline
    W & 2019 & 3 & B & Programación & 1035 & 12 & Base de Datos  \\
    \hline
    \end{tabular}
    \caption{Ejemplo de una planilla de datos de un plan de estudios}
    \label{tab:tabla_planes}
\end{table}

\subsection[Datos personales]{Datos personales}

La planilla de datos personales contiene la información con el formato detallado en la Tabla \ref{tab:tabla_datos}.

\begin{table}[!htbp]
    \centering
    \begin{tabular}{|c|c|c|c|c|c|c|c|}
    \hline
    Legajo & DNI & Apellido & Nombre & Email & Fecha & Carrera & Plan \\ \hline
    21872 & 35905769 & Yegro & Juan & jy@unq.edu.ar & 01-02-2009 & W & 2019 \\
    \hline
    \end{tabular}
    \caption{Ejemplo de una planilla de datos personales de los estudiantes}
    \label{tab:tabla_datos}
\end{table}

Estos datos son útiles para que los directores puedan identificar a los estudiantes.

\subsection[Materias cursadas]{Materias cursadas}

La planilla del historial de cursadas de una carrera determinada tiene el siguiente formato, donde cada fila representa la cursada de un estudiante en una materia determinada.

\begin{outline}
    \2 Legajo.
    \2 DNI 
    \2 Carrera 
    \2 Regular 
    \2 Calidad 
    \2 Materia 
    \2 Nombre 
    \2 Fecha 
    \2 Resultado 
    \2 Nota 
    \2 Forma 
    \2 Créditos
    \2 Acta 
    \2 Acta E. 
    \2 Plan
\end{outline}

\subsection[Inscripciones]{Inscripciones}

Esta planilla indica las inscripciones a una materia por parte de un estudiante.

\begin{table}[!htbp]
    \centering
    \makegapedcells
    \begin{tabular}{|c|c|c|c|c|c|}
    \hline
    Carrera & DNI & Legajo & Código de materia & Comisión & Fecha  \\\hline
    W & 35905769 & 21872 & 01036 & C2 & 01/03/2016  \\
    \hline
    \end{tabular}
    \caption{Ejemplo de una planilla de inscripciones de estudiantes}
    \label{tab:tabla_datos}
\end{table}


\section[Trabajos relacionados]{Trabajos relacionados}

\subsection[eCoach]{eCoach}
ECoach es una aplicación diseñada en la Universidad de Michigan para estudiantes de primer año que toman clases de ciencia, tecnología, ingeniería o matemática. Esta aplicación toma datos que las universidades ya tienen, como las materias que cursó un estudiante, las notas, etc.
Otro aspecto importante de esta aplicación, es que se recibe \textit{feedback} de parte de los estudiantes con respecto a materias, profesores, horarios, etc.

\subsection[Planteamiento del sistema de gestión académica orientado a serverless y microservicios]{Planteamiento del sistema de gestión académica orientado a serverless y microservicios}

El trabajo “Planteamiento del sistema de gestión académica orientado a serverless y microservicios de la Universidad Distrital F.J.D.C
orientado a server less y micro servicios” (3) fue desarrollado en la Universidad Distrital de Universidad Distrital Francisco José de Caldas de Bogotá (Colombia). Este trabajo contempla el desarrollo de una aplicación que implica diferentes aspectos: la asignación a y evaluación de docentes en las materias, generación de reportes de evolución, gestión de movilidad de estudiantes de intercambio, manejo de notas, inscripciones comunes y basadas en tutorías, gestión de tutorías, administración de planes de estudios, de exámenes y de trabajos finales. El alcance del trabajo es muy amplio si se compara con este trabajo. La diferencia es que solamente se presentan los modelos de las arquitecturas, sin una implementación concreta del mismo.


\subsection[Sistema de Seguimiento Académico de Alumnos Universitarios, Universidad Nacional de Formosa]{Sistema de Seguimiento Académico de Alumnos Universitarios, Universidad Nacional de Formosa}
El trabajo “Sistema de Seguimiento Académico de Alumnos Universitarios”  de la materia Programación III de la Universidad Nacional de Formosa. Es un proyecto que tiene como objetivo capturar el rendimiento académico y la situación en la que se encuentra cada estudiante, a partir de un puntaje calculado por un sistema mediante los datos almacenados en una base
de datos diseñada especialmente para el sistema en sí. El objetivo final es detectar a aquellos estudiantes con mayor probabilidad de dejar la carrera, de acuerdo al puntaje obtenido, o con mayor probabilidad de que se “estanque”; debido a fallos en el transcurso de la misma. El cálculo se realiza en base a las siguientes variables: antecedenes académicos, información del desempeño actual de los estudiantes, antecedentes socio-económicos y personales y razones de disersión. En el caso particular de este trabajo, es similar a lo planteado, pero solo busca un índice para la evaluación de los estudiantes. En nuestro trabajo, buscamos que la aplicación genere diferentes índices y resultados que ayuden a diversas decisiones que puede tomar la
Dirección de las carreras de la UNQ.

\section[Análisis de requerimientos]{Análisis de requerimientos}

Luego de hacer un análisis de requerimientos con directores de las distintas carreras de Ciencia y Tecnología de la Universidad Nacional de Quilmes, se identificaron las siguientes necesidades:

\begin{outline}
    \1 Calcular el avance de carrera, avance de diplomatura y avance de ciclo introductorio en base a los créditos.
    \1 Calcular el porcentaje de materias aprobadas del total de la Carrera, el porcentaje de materias del Ciclo Básico, el porcentaje de materias del Ciclo Avanzado, el porcentaje de las Complementarias y el porcentaje del Ciclo Introductorio.
    \1 Calcular el porcentaje de aprobación por área, que en el caso de la Licenciatura en Informática estas áreas son: 
        \2 Programación.
        \2 Sistemas Informáticos.
        \2 Procesos Informáticos.
        \2 Desarrollo de Software.
        \2 Teoría de la Computación.
    \1 Calcular los pedidos específicos de CONEAU para acreditación de carreras:
        \2 Postulantes: cantidad de estudiantes que se anotaron.
        \2 Ingresantes: cantidad de estudiantes que cursaron al menos una materia del Ciclo Introductorio.
        \2 Estudiantes: Los que no son ingresantes. Estudiantes activos.
        \2 Tabla de ingresantes por cohorte.
        \2 Tabla de cursantes por cohorte.
        \2 Tabla de graduados por cohorte.
        \2 Tasas de crecimiento anual y por cuatrimestre de las carreras, cursantes y solo ingresantes.
    \1 Para una materia determinada conocer:
        \2 Cantidad de aprobados.
        \2 Cantidad de ausentes.
        \2 Cantidad de desaprobados.
    \1 Calcular listados de estudiantes con una determinada materia aprobada, que distinga en: materia aprobada por equivalencia, examen libre, pendiente, acta de cursada común.
    \1 Para los inscriptos a una materia generar listados de cuáles son recursantes, y cuántas veces.
    \1 Realizar análisis retrospectivos y riesgos de deserción en base al historial general de la carrera.
    \1 Identificar cuáles son las materias que demoran en la formación a los estudiantes en las carreras. 
    \1 Predecir inscripciones a carreras y materias.
    \1 Analizar cumplimientos de prerrequisitos según plan de estudios.
        
\end{outline}


\section[Conclusiones]{Conclusiones}

Al implementar esta solución, los datos estarán disponibles para ser consultados por las carreras. Además, existe un potencial para que esta solución pueda ser extendida en el futuro y que esto implique una mejor experiencia también para los estudiantes.
Llevar a cabo este proyecto implicará una mejora en cuanto a decisiones que se pueden tomar con respecto a los estudiantes en pos de una mejor calidad educativa. Ayudará a determinar qué estudiantes están teniendo problemas, qué materias están trabando a los estudiantes, cuáles son los estudiantes que corren riesgo de deserción, cuáles están cerca de recibirse, entre otras decisiones relevantes.



